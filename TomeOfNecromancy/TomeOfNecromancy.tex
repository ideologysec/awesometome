\documentclass[10pt]{report}

\usepackage{appendix}
\usepackage{floatflt}
\usepackage{fancyhdr}
\usepackage{textcomp}
\usepackage[usenames]{color}
\usepackage{isoent}    %\sfrac
%\usepackage{palatino} %font
\usepackage{newcent}   %font
\usepackage{sectsty} %custom section headings

\newcommand{\normalsections}{
\sectionfont{\noindent\rule{\textwidth}{0.015in}\\\nohang}
\subsectionfont{\noindent\rule{\textwidth}{0.005in}\\\nohang}
}

\newcommand{\columnsections}{
\sectionfont{\vspace*{-20pt}\noindent\rule{3.5in}{0.015in}\\\nohang}
\subsectionfont{\vspace*{-20pt}\noindent\rule{3.5in}{0.005in}\\\nohang}
}


\sectionfont{\noindent\rule{\textwidth}{0.015in}\\\nohang}
\subsectionfont{\noindent\rule{3.5in}{0.005in}\\\nohang}

\usepackage[Bjarne]{fncychap} %Canned chapter headings


\usepackage{multicol}
\usepackage[bookmarks=true,colorlinks,linkcolor=cyan,breaklinks]{hyperref}



%%%%Margins%%%
\topmargin 0pt
\advance \topmargin by -\headheight
\advance \topmargin by -\headsep
\textheight 8.9in
\oddsidemargin -0.25in
\evensidemargin \oddsidemargin
\textwidth 7in
\oddsidemargin -0.25in
%\setlength {\parindent} {0pt}



%%%Formatting%%%
\newcommand{\ability}[2]{\smallskip \noindent \textbf{#1} #2}
\newcommand{\shortability}[2]{\noindent\textbf{#1} #2\\}
\newcommand{\bolded}[1]{\noindent\textbf{#1}}
\newcommand{\itemability}[2]{\item \textbf{#1} #2}
\newcommand{\featname}[1]{\vspace*{0.1cm plus 0.2cm minus 0.05cm}\noindent\textbf{#1}\\}
\newcommand{\featnamelist}[1]{\vspace*{0.1cm plus 0.2cm minus 0.05cm}\noindent\textbf{#1}}

\newcommand{\descfeat}[2]{\featname{#1}\emph{#2}\\}
\newcommand{\business}[6]{
	\vspace{2pt}
	\noindent\hspace*{1cm}\textbf{#1}\\
	\noindent\hspace*{1cm}\textbf{Primary Skill:} #2\\
	\noindent\hspace*{1cm}\textbf{Secondary Skills:} #3\\
	\noindent\hspace*{1cm}\textbf{Capital:} #4\\
	\noindent\hspace*{1cm}\textbf{Resources:} #5\\
	\noindent\hspace*{1cm}\textbf{Risk:} #6
	\vspace{2pt}
}
\newcommand{\classname}[1]{\section{#1}}
\newcommand{\condition}[1]{\emph{#1}}
\newcommand{\quot}[1]{\emph{#1}\medskip}
\newcommand{\desc}[1]{#1 \medskip}
\newcommand{\example}[1]{\emph{#1}}
\newcommand{\magicitem}[1]{\emph{#1}}
\newcommand{\monster}[1]{\subsection{#1} \label{monster:#1}}
\newcommand{\monsterline}[2]{\textbf{#1:} #2\\}
\newcommand{\monstersizetype}[2]{\textbf{#1 #2}\\}
\newcommand{\spell}[1]{\emph{#1}}
\newcommand{\spelllist}[1]{\smallskip \noindent \underline{\textbf{#1}}}

% For the feats -- moved here from feats.tex because we added another 
% section(s) for feats. -Surgo
\newcommand{\minitabular}[1]{\begin{tabular}{p{0.25in}p{2.9in}} #1\\ \end{tabular}}
\newcommand{\babfeat}[7]{
	\noindent\minitabular{\multicolumn{2}{l}{\parbox{3in}{\textbf{#1}}}}
	\minitabular{\multicolumn{2}{l}{\parbox{3in}{\small #2}}}
	\minitabular{\raggedleft\textbf{\small \textbf{+0:}}& {\small #3}}
	\minitabular{\raggedleft\textbf{\small +1:} & {\small #4}}
	\minitabular{\raggedleft\textbf{\small +6:} & {\small #5}}
	\minitabular{\raggedleft\textbf{\small +11:} & {\small #6}}
	\minitabular{\raggedleft\textbf{\small +16:} &{\small #7}}
}
\newcommand{\skillfeat}[8]{
	\noindent\minitabular{\multicolumn{2}{l}{\parbox{3in}{\textbf{#1}}}}
	\minitabular{\multicolumn{2}{l}{\parbox{3in}{\small #2}}}
	\minitabular{\multicolumn{2}{l}{\parbox{3in}{\small\textbf{#3}}}}
	\minitabular{\raggedleft\textbf{\small \textbf{0:}} &{\small #4}}
	\minitabular{\raggedleft\textbf{\small 4:} &{\small #5}}
	\minitabular{\raggedleft\textbf{\small 9:} &{\small #6}}
	\minitabular{\raggedleft\textbf{\small 14:} &{\small #7}}
	\minitabular{\raggedleft\textbf{\small 19:} &{\small #8}}
}

\newcommand{\boxit}[1]{\frame{\parbox{\textwidth}{#1}}}

% A new box command -- the first argument is the box's header, the second the 
% contents of the box. This looks a lot better than \boxit. -Surgo
\newcommand{\abox}[2]{\vspace{5pt}
	\fbox{\begin{minipage}{0.8\linewidth}
	\setlength{\parindent}{0.15in}
	\textbf{#1}

	#2
\end{minipage}}
\vspace{5pt}}

\newcommand{\itemspace}{\setlength{\itemsep}{-1mm}\setlength{\topsep}{-1mm} }

\newcommand{\listone}{\begin{list}{$\bullet$}{\itemspace}}
\newcommand{\listprereq}{\begin{list}{\vspace*{-2pt}}{\itemspace}}
\newcommand{\listtwo}{\begin{list}{$\triangleright$}{\itemspace}}
\newcommand{\listthree}{\begin{list}{--}{\itemspace}}

% Bold the first argument in the listitem
\newcommand{\bolditem}[2]{\item \textbf{#1} #2}

\newcommand{\slashfrac}[2]{${}^{#1}$\hspace*{-1pt}\makebox[2pt]{$\diagup$}${}_{#2}$}
%\newcommand{\half}[0]{\slashfrac{1}{2}}
\newcommand{\half}[0]{\ensuremath{\sfrac{1}{2}} }
\newcommand{\third}[0]{\ensuremath{\sfrac{1}{3}} }
\newcommand{\fourth}[0]{\ensuremath{\sfrac{1}{4}} }
%\newcommand{\half}[0]{\ensuremath{\scriptscriptstyle 1/2}}
%\newcommand{\threefourths}[0]{\ensuremath{\sfrac{3}{4}}}


%\rule{\textwidth}{0.005in}

%\renewcommand{\sectionmark}[1]{\markright{\thesection\ \boldmath\emph{#1}\unboldmath}\\\rule{\textwidth}{0.005in}\\}

\begin{document}

%%%%%%% Front several pages
\pagestyle{plain}
%Cover Page

\begin{center} \Huge
\textsc{ClasslessD20 Modern}
\end{center}

\newpage


\pagestyle{fancy}
%Fix the spacing so it's all reasonable
\linespread{.9}  \small  \normalsize \itemspace \normalsections

\tableofcontents

\chapter*{Foreword}

Unlike the Revised Necromancer Handbook, which is a compilation of the Necromancy rules as they stand, what you are reading now is the rules for Necromancy as they should be. We feel there is a need for this because despite (or let's not kid ourselves, because of) the considerable amount of space spent given over to Necromancy in officially sanctioned products, the classical Necromancer does not function under the rules as written. Vampires can't run or be staked, there aren't any prestige classes that make you any more of a necromancer than you are with the base classes, and honestly no one even knows how the basic necromancy spells work. Not because they are stupid, but because the rules for such things are contradictory in several key places.

\input{morality}

%%%%%%% Base Classes
\chapter{Necromancy with Class}

The degree to which the published classes of Necromancy aren't good causes people physical pain. The degree of malarkey that people are willing to attempt in order to use these classes in a half-way level appropriate way causes \emph{us} physical pain. While the flavor of published necromantic classes is frequently adequate or even engaging -- the mechanics just aren't there. You shouldn't have to cheat just to make a concept character \emph{nearly} the equal of a standard Cleric or Wizard. As a solution, we propose having actually mechanically viable prestige classes for the necromantically inclined to use:

\input{prestige/corpselight}
\input{prestige/uttercold}
\input{prestige/bonebladereaper}
\input{prestige/bonebladereaper_revised}
\input{prestige/skindancer}
\input{prestige/strangerwithburningeyes}
\input{prestige/masterofnecromanticmysteries}
\input{prestige/deathking}
\input{prestige/widowqueen}
\input{prestige/bonerider}
\input{prestige/thiefofsouls}
\input{prestige/soulmerchant}
\input{prestige/lurker}
\input{prestige/heartlessmage}
\input{prestige/speakerfordead}
\input{prestige/lorddamned}
\input{prestige/pumpkinking}

\chapter{Necromancers with Style}

\begin{multicols}{2}

\hypertarget{feat:weaponrighteous}{}\babfeat{Weapon of Righteous Destruction [Combat]}{

Your hands make whatever is being held by them holy and on fire. For some reason this doesn't make them melt or burn up.}{

Whatever weapon you are wielding is considered Magical (+1/3 bonus/level) in addition to any other properties that it has. Your unarmed attacks, even if not proficient, count for this effect.}{

The above, Flaming weapon.}{

The above, Holy instead of Flaming.}{

The above, Sun weapon, Fort save. (BoG)}{

The above, Vorpal weapon (BoG).}

The following feats from the Tome of Fiends are now also [Celestial] feats, and you don't pick up fiendish traits for taking them as a [Celestial] feat:

Breath Weapon 

Elemental Aura 

Extra Arms 

Extra Summons 

Greater Teleport 

Harmless Form 

Large Size 

Huge Size 

Heighten Spell Like Ability 

\descfeat{Product of Celestial Dalliance}{One of your recent ancestors was an celestial Outsider or from a good-aligned plane. Maybe your parents play it off as a virgin birth, maybe your dad became a Saint.}
You may take any [Celestial] feat, and gain Resistance 5 to Acid, Cold, Electricity. You also gain one of the Angel, Archon, Eladrin, or Guardinal subtypes, and a Smite Evil attack usable at will that does bonus damage equal to 1/2 of your strength modifier.

\featname{Wings of Good [Celestial]}
You gain wings and a fly speed equal to double your base land speed with good maneuverability that requires special armor to stay aloft. These must be feathery or energy-based.

\hypertarget{feat:lordofdeath}{}\skillfeat{Lord of Death [Necromantic] [Skill] [Leadership]}{
A whole bunch of skeletons and crap show up to fight under your tattered banner.}{
Knowledge (Religion) Ranks:}{
You have a Command Rating equal to your ranks in Knowledge Religion divided by five (round up). You are a Necromantic leader (see Heroes of Battle).}{
You can muster a group of followers. Your leadership score is your ranks in Knowledge Religion plus your Wisdom modifier. Your followers are all mindless Undead. You don't make them or anything, they just show up.}{
You are able to delegate command to a loyal cohort. Your cohort is an intelligent and loyal Undead creature with a CR at least 2 less than your character level. Cohorts gain levels when you do.}{
Your followers swell in number to that of an army.}{
Your allies gain energy resistance to Positive Energy equal to your level while they are within line of sight of you.}

\hypertarget{feat:tyrant}{}\skillfeat{Tyrant [Skill] [Leadership]}{
You push people around and get larger and larger groups trapped in the iron gauntlet of your brutal rule.}{
Intimidate Ranks:}{
You inspire such terror that creatures you intimidate continue to act intimidated after you leave, too afraid to raise their voice in defiance even after you have apparently left them far behind.}{
You can muster a group of followers. Your leadership score is your ranks in Intimidate plus your Strength modifier.}{
Your followers swell in number to that of an army.}{
Your mere presence inspires fear and can break a battle. Enemies with more than 5 hit dice less than you do must make a Will save (DC 10 + � Level + Strength Modifier) of flee in \emph{panic}.  This is a [Fear] effect.}{
Your presence causes despair in even brave opponents. All enemies within 30' of your suffer a -2 Morale penalty to Willpower saves.}

\hypertarget{feat:monsterrancher}{}\skillfeat{Monster Rancher [Skill] [Leadership]}{
You can breed and train a large number of crazy beasts.}{
Handle Animal Ranks:}{
You can use Handle Animal as if it were Diplomacy when dealing with Magical Beasts and Dragons. You can do similarly with Aberrations and Plants with an Intelligence Score that is less than 9.}{
You can muster a group of followers. Your leadership score is your ranks in Handle Animal plus any synergy bonuses you gt to that skill. Your followers can, and must be monsters.}{
You have a loyal cohort that is a monster of some kind. A cohort is an intelligent and loyal creature with a CR at least 2 less than your character level. Cohorts gain levels when you do.}{
You know what any monster is unless it is disguised by illusion, and you can look up its stat line in the appropriate monster book when devising your strategies.}{
Once per day, you can reroll a saving throw allowed by a Supernatural Ability.}

\hypertarget{feat:armyofdemons}{}\skillfeat{Army of Demons [Skill] [Fiend] [Celestial] [Leadership]}{
You have an army of planar crazy crap.}{
Knowledge (Planes) Ranks:}{
You have a Command Rating equal to your Knowledge: Planes ranks divided by five (round up).}{
You can muster a group of followers. Your leadership score is your ranks in Knowledge: Planes plus your Charisma mod. These followers can and must be outsiders.}{
Your followers swell in number to that of an army.}{
You own a planar stronghold.}{
Your allies gain a +2 morale bonus to all saving throws if they can see you and you are within medium range.}

\hypertarget{feat:bureaucrat}{}\skillfeat{Bureaucrat [Skill] [Leadership]}{
You have a functioning guild that makes stuff for you and gives you money.}{
Appraise Ranks:}{
You draw an income for working as an administrator, getting 1 GP/week per rank in Appraise.}{
You can muster a group of followers. Your leadership score is your ranks in Appraise plus your Intelligence modifier. These followers all have profession and craft skills.}{
You get your own Stronghold.}{
You get a +2 bonus to profit checks.}{
Your guild goes planar, your number of followers swell to the size of an army and their ranks start filling up with producers and managers from other planes of existence.}

\end{multicols}

\chapter{The Necronomicon}

Necromancy as a school is possessed of some of the most powerful and game defining spells ever imagined in the worlds of Dungeons and Dragons. \spell{Magic jar}, \spell{wail of the banshee}, and \spell{clone} can practically be a world threatening plan for a BBEG all by themselves. In fact, that's been done several times. But Necromancy as a school suffers greatly for this attention. Though the earth shaking power for dark lords is well represented, the low levels of necromancy have been largely ignored by generations of authors. It is our intention to produce a short list of spells that allow a low level Necromancer to be memorable and effective without constantly falling back on the old stand-by of having Spell Focus: Conjuration.

\section{New Spells}

\subsection{Some Spells Don't Work}

Many spells are underwhelming for their level or have mechanics that are hard to explain. But first and foremost of all the spells that are bad for the game is Polymorph. That spell is integral to any fantasy setting, but people haven't made it work in 3rd edition. Mostly, this is because people keep writing it long instead of short. Remember, if you can't explain an effect in 2 minutes, everyone else is already confused.

\subsection{Necromancy}

Necromancy as a school is possessed of some of the most powerful and game defining spells ever imagined in the worlds of Dungeons and Dragons. \spell{Magic Jar}, \spell{Wail of the Banshee}, and \spell{Clone} can practically be a world threatening plan for a BBEG all by themselves. In fact, that's been done several times. But Necromancy as a school suffers greatly for this attention. Though the earth shaking power for dark lords is well represented, the low levels of necromancy have been largely ignored by generations of authors. It is our intention to produce a short list of spells that allow a low level Necromancer to be memorable and effective without constantly falling back on the old stand-by of having Spell Focus: Conjuration.

The [Healing] subschool:
The spell \spell{cure light wounds} has no business being in the school of Conjuration. It's not that you can't make an acceptable argument for the existence of ``conjuration" that makes people feel better -- that's actually pretty easy to rationalize. It's that \spell{cure light wounds} doesn't work the way a spell that was in Conjuration would work. It doesn't create healthy flesh to fill up wounds -- it channels Positive Energy into the creature and makes them feel better or worse depending upon how they react to that sort of thing. As described, the [Healing] subschool needs to be in the same school as \spell{inflict light wounds}, because it does the same thing. Logically speaking, that could be Evocation (because Evocation handles any Energy Channeling), or it could be Necromancy (because Necromancy can do pretty much anything with Positive or Negative Energy). We suggest having the [Healing] subschool in Necromancy, but only because this isn't The Tome of Evocation. If you decide to make these spells Evocation spells for your home game, we won't stop you.\\

\begin{quote}

\featname{Congealing Consumption}
\begin{small}
\shortability{Necromancy}{}
\shortability{Level:}{Sor/Wiz 2}
\shortability{Components:}{V, S}
\shortability{Casting Time:}{1 Standard Action}
\shortability{Range:}{Medium}
\shortability{Area:}{10 foot radius burst.}
\shortability{Duration:}{Instantaneous, and 1 round/level (see below)}
\shortability{Saving Throw:}{Willpower Negates}
\shortability{Spell Resistance:}{Yes.}
\end{small}
\emph{As the necromancer finishes the final incantations, a dark cloud arises and envelopes the souls of those within.}

Any creature within the area when the spell is cast must make a Willpower save or be nauseated for one round per level of the caster.\\

\featname{Curse of Crumbling Conviction}
\begin{small}
\shortability{Necromancy}{}
\shortability{Level:}{Sor/Wiz 4}
\shortability{Components:}{V, S}
\shortability{Casting Time:}{1 Standard Action}
\shortability{Range:}{Medium}
\shortability{Target:}{One Creature.}
\shortability{Duration:}{Instantaneous}
\shortability{Saving Throw:}{Willpower Negates}
\shortability{Spell Resistance:}{Yes.}
\end{small}
\emph{The avenging angel glared down with menace at the necromancer. She raised her flaming sword even as he completed his spell. He met her smoldering gaze levelly. ``Why?" he asked. It was a question she could not answer\ldots}

If the target fails their save they no longer feel strongly about people, ideals, or things. The target's alignment becomes neutral if it wasn't already, and the creature becomes indifferent to everyone, including the caster. This effect is an instantaneous shift, and in no way prevents a creature from responding to subsequent diplomacy or threats. This lack of purpose is an oppressive feeling for intelligent creatures, who will gladly adopt the alignment of whatever creature next persuades them to being helpful. Creatures with an alignment subtype will gradually find their purpose again -- regaining the alignment of their subtype in a d4 days unless they already have a new one.\\

\featname{Dark Symmetry}
\begin{small}
\shortability{Necromancy}{}
\shortability{Level:}{Sor/Wiz 2}
\shortability{Components:}{S}
\shortability{Casting Time:}{1 Standard Action}
\shortability{Range:}{Medium}
\shortability{Target:}{One Creature.}
\shortability{Duration:}{Concentration}
\shortability{Saving Throw:}{Willpower Negates}
\shortability{Spell Resistance:}{Yes.}
\end{small}
\emph{Her hands slow down from the frenzied pace and the necromancer's shadow extends into the target's. The warrior's sword arm slows and holds fast. A smile flashes across her face, and she takes a step forward. The warrior unsteadily takes a step backwards, a look of panic crossing his face.}

If the victim fails their saving throw, they are helpless and unable to voluntarily move until the spell is terminated. Further, if the caster moves while concentrating upon the spell, the victim simultaneously moves an equal distance in the same direction. If the victim is moved into an occupied space, he falls prone. If the victim is moved off a cliff, he falls.\\

\featname{Form of Death}
\begin{small}
\shortability{Necromancy}{}
\shortability{Level:}{Sor/Wiz 2}
\shortability{Components:}{S}
\shortability{Casting Time:}{1 Standard Action}
\shortability{Range:}{Touch}
\shortability{Target:}{One Living Creature.}
\shortability{Duration:}{24 hours}
\shortability{Saving Throw:}{Fortitude Negates (Harmless)}
\shortability{Spell Resistance:}{Yes (Harmless).}
\end{small}
\emph{The final incantations completed, the necromancer's skin turned gray, his lips became cold and dry.}

A creature affected by Form of Death becomes very much like an undead creature. The target gains [Undead] as a subtype and is affected by any spell or effect that targets Undead specifically. The target is also cured by negative energy and damaged by positive energy. The target is immune to negative energy levels, ability damage, and ability drain. The character will be treated as an undead by those around him, which is both a boon and a bane -- while mindless undead won't attack the target unless specifically ordered to and the character gains a +4 profane bonus on all charisma related checks when used on undead creatures, the target also suffers a -4 penalty on Charisma related checks for dealing with living creatures.\\

\featname{Puppet Dance}
\begin{small}
\shortability{Necromancy}{}
\shortability{Level:}{Sor/Wiz 3}
\shortability{Components:}{S}
\shortability{Casting Time:}{1 Standard Action}
\shortability{Range:}{Close}
\shortability{Target:}{One Corporeal Creature.}
\shortability{Duration:}{Concentration}
\shortability{Saving Throw:}{Reflex Negates}
\shortability{Spell Resistance:}{Yes.}
\end{small}
\emph{The necromancer holds her hands up with fingers apart, shadowy tendrils hang down from each finger, tapering into nonexistence before reaching her waist. As the spell reaches completion, larger tendrils appear above the target and hang down to anchor themselves in the victim's flesh.}

If the victim fails their saving throw, they are helpless and unable to move voluntarily until the spell is terminated. This spell only affects creatures with a physical body. When the caster spends a standard action to concentrate on the spell, she may opt to have the victim move and perform a physical standard action. The caster cannot force the victim to use their spell knowledge (if any), and any attacks made by the victim use the caster's Base Attack Bonus rather than their own.\\

\featname{Sobering Skeletal Stillness}
\begin{small}
\shortability{Necromancy}{}
\shortability{Level:}{Sor/Wiz 1}
\shortability{Components:}{V, S}
\shortability{Casting Time:}{1 Standard Action}
\shortability{Range:}{Medium}
\shortability{Target:}{One Creature with Bones.}
\shortability{Duration:}{Concentration}
\shortability{Saving Throw:}{Fortitude Negates}
\shortability{Spell Resistance:}{Yes.}
\end{small}
\emph{Chortling like a man possessed, the necromancer contorts his hands into unnatural positions, emitting dreadful crackling sounds of bones grinding against one another. A black aura surrounds his victim, and the sounds of crepitace now come from two\ldots}

If the victim fails their saving throw, they are helpless and unable to move until the spell is terminated. This spell only affects creatures who have a skeletal structure, although an exoskeleton does count. Creatures normally immune to paralysis, necromantic effects, or effects requiring a fortitude save that do not affect objects are still affected by this spell if they have a skeleton (so a zombie ogre is affected, but an iron golem is not).\\

\featname{Tasha's Tomb Tainting}
\begin{small}
\shortability{Necromancy}{}
\shortability{Level:}{Sor/Wiz 1}
\shortability{Components:}{V, S, M}
\shortability{Casting Time:}{10 minutes}
\shortability{Range:}{Touch}
\shortability{Area:}{40-ft. radius emanating from the touched point}
\shortability{Duration:}{Instantaneous}
\shortability{Saving Throw:}{See text}
\shortability{Spell Resistance:}{No.}
\end{small}
\emph{The graveyard has a serene feeling; the dead have lain here undisturbed for generations. Above the gate a sign in ancient Dwarven earnestly proclaims that the heroes interred within shall lie in peace forever. The necromancer puzzles through the faint and archaic runes and chuckles to himself. ``Not hardly?" he mutters, and begins fishing through his spell component pouch for a black pearl\ldots}

Upon spell completion, the area is free of any consecration, desecration, Forsaken Graveyard, hallow, Tomb, or unhallow effects. If the caster chooses, the area can be considered desecrated for the next 24 hours (the effects are increased as if there had been a permanent altar to an evil god or pantheon in the area).

\shortability{Material Component:}{One Black Pearl, worth at least 500 gp.}

\featname{Tasha's Tomb Transport}
\begin{small}
\featname{Necromancy}
\shortability{Level:}{Sor/Wiz 3}
\shortability{Components:}{V, S}
\shortability{Casting Time:}{1 minute}
\shortability{Range:}{See Text}
\shortability{Target:}{You and objects and willing creatures.}
\shortability{Duration:}{Instantaneous}
\shortability{Saving Throw:}{See text}
\shortability{Spell Resistance:}{No.}
\end{small}
\emph{The caster finishes the droning incantations and places her hand on the ground, where dark red runes appear in a circle around it. Eerily dark tendrils rise from the shadows and consume everyone and everything within the area. Far away, a black portal opens on the ground and the travelers rise from it covered in a thin sheen of cold sweat.}

This spell can only be cast in a Tomb, and it transports the targets to another Tomb of the caster's choice. The caster has no ability to determine where in the Tomb she will end up, but all targets appear together. The target Tomb need not be on the same plane of existence, but the caster must know where the target Tomb is to within one mile. The spell fails if either Tomb is cut off from the Negative Energy Plane (including effects like dimensional interdiction). Total transported creatures and objects cannot exceed 500 pounds per caster level in weight.\\

\featname{Tomb Tile Tessellation}
\begin{small}
\shortability{Necromancy}{}
\shortability{Level:}{Sor/Wiz 2}
\shortability{Components:}{V, S, M}
\shortability{Casting Time:}{10 minutes}
\shortability{Range:}{Touch}
\shortability{Area:}{One ten-foot cube}
\shortability{Duration:}{Instantaneous}
\shortability{Saving Throw:}{No.}
\shortability{Spell Resistance:}{No.}
\end{small}
\emph{Who knew how long the dead had lain in repose, unmolested and forgotten? Judging by the state of the remains, it was long enough -- so the necromancer brought out the charts and began planning out how to make sure it stayed that way.}

This spell can only be cast in a place where the dead have lain for at least 50 years without being returned as undead. Each casting makes one ten foot cube eligible to become a Tomb. Once the entire area is eligible to become a Tomb, the entire area becomes a Tomb.

\shortability{Material Component:}{One vial of Holy Water or one vial of Unholy Water.}
\end{quote}


\subsection{Polymorph}

\subsubsection{Polymorph Version 1: Character Replacement}

If you take part of your character -- any part of your character -- and part of a monster from one of the many monster books in D\&D, and you put them together into a single Voltron-like body, you have broken D\&D. That should be obvious, but since we are over six years into the ridiculous circus that is polymorph in 3rd edition Dungeons and Dragons, apparently it isn't. If it is important to you that you be allowed to dumpster dive through the monster books and find an appropriate to transform into, it is important to D\&D that absolutely no part of your character be mixed and matched during that period. If you want to truly become a monster, you have to actually become that monster. Not ``the monster with all my spell effects running", not ``the monster with my formidable mental attributes. No. You need to become the monster exactly as it appears in the monster book or there's no chance of you getting a balanced result. Some people are going to end up as mediocre monsters with carry-over abilities that happen to synergize well and become tremendously powerful while other people are just as unbalanced in the other direction when they find that drawbacks of their character are carried over and overwrite the abilities of a monster that are supposed to make them any good at all.

And this isn't just hyperbole or doomsday predictions, this is established fact. We've all played with some of the multitude of different versions of Polymorph errata and ``fixes", and the abject horror caused by every single iteration. The idea doesn't work. If you're going to replace any part of the character, you have to replace it all. So here's a version of polymorph that won't make us cry. This ain't rocket science, it just takes a little bit of discipline:

\begin{quote}
\featname{Polymorph Self}
\begin{small}
\shortability{Transmutation}{}
\shortability{Level:}{Sor/Wiz 4}
\shortability{Components:}{V, S}
\shortability{Casting Time:}{1 Standard Action}
\shortability{Range:}{Self}
\shortability{Duration:}{10 minutes/level (D)}
\shortability{Saving Throw:}{Fortitude Negates (Harmless)}
\shortability{Spell Resistance:}{No}
\end{small}
\quot{``A Turtle am I? Let's see how Turtlike I\ldots\  CAN\ldots\  BE!'' And with that, the mage was a giant turtle.}

You vanish and a monster of your choice appears in your place. The creature shares your alignment, personality and goals, and will continue to act as you would within the limits of its intelligence and abilities. The creature must be at least 3 CR less than your character level, may not have the incorporeal or swarm subtype, and is unexceptional for its type. If the monster is killed, the spell is ended. When the spell ends, the monster vanishes and you appear where the monster was with an amount of lethal, nonlethal, and ability damage on you equal to the amount the monster had suffered when the spell ended (this means that if the spell ended because the monster was slain and the monster had an equal or greater number of hit points as you, you may well be dead when you appear).
\end{quote}

\begin{quote}
\featname{Polymorph Other}
\begin{small}
\shortability{Transmutation}{}
\shortability{Level:}{Sor/Wiz 4}
\shortability{Components:}{V, S}
\shortability{Casting Time:}{1 Standard Action}
\shortability{Range:}{Medium}
\shortability{Target:}{One Creature}
\shortability{Duration:}{Permanent (D)}
\shortability{Saving Throw:}{Fortitude Negates}
\shortability{Spell Resistance:}{Yes}
\end{small}
\quot{The witch snarled at the trespasser and pointed her wand vindictively at him. A short incantation later left nothing but a pig in his place.}

Your target vanishes and a creature of your choice appears in its place. The creature shares the alignment, personality and goals of the target, and will continue to act as it would within the limits of its intelligence and abilities. The creature must be at least 5 CR less than your character level, may not have the incorporeal or swarm subtype, and is unexceptional for its type. If the creature is killed, the spell is ended. When the spell ends, the creature vanishes and the target appears where the creature was with an amount of lethal, nonlethal, and ability damage on it equal to the amount the creature had suffered when the spell ended (this means that if the spell ended because the creature was slain and the creature had an equal or greater number of hit points as the original target, it may well be dead when it appears).

\end{quote}


\begin{quote}
\featname{Mass Polymorph}
\begin{small}
\shortability{Transmutation}{}
\shortability{Level:}{Sor/Wiz 7}
\shortability{Components:}{V, S}
\shortability{Casting Time:}{1 Standard Action}
\shortability{Range:}{Medium}
\shortability{Target:}{Any number of creatures within a 20' radius}
\shortability{Duration:}{Permanent (D)}
\shortability{Saving Throw:}{Fortitude Negates}
\shortability{Spell Resistance:}{Yes}
\end{small}
\quot{The crowd looked uncomfortable. They had weapons and were brandishing them in a fashion quite menacing. But the magician was laughing, and that really put a damper on the mood of the entire event. They started to regain their composure and again advance upon him. He snorted and muttered an incantation, and something about swine \ldots}

Each target vanishes and creature of your choice appears in its place. The creatures share the alignment, personality and goals of the targets, and will continue to act as they would within the limits of their intelligence and abilities. The creatures must be at least 7 CR less than your character level, need not be the same for all targets, none may have the incorporeal or swarm subtype, and all are unexceptional for their type. If a creature is killed, the spell is ended for that target only. When the spell ends, the creatures vanish and the targets appear where the creatures were with an amount of lethal, nonlethal, and ability damage on it equal to the amount the creature had suffered when the spell ended (this means that if the spell ended for a target because the creature was slain and the creature had an equal or greater number of hit points as the original target, it may well be dead when it appears).

\end{quote}

\subsubsection{Polymorph Version 2: Fixed Forms}

The other version is one where transforming leaves you essentially yourself, only with a new hairdo and possibly some bonuses. In this case, you keep everything about yourself and simply get a disguise and some advantages consistent with a buff spell. All of the ``Whatever-Form" spells don't stack with multiple castings or even with each other, because they are considered to be ``one spell makes another spell irrelevant" for purposes of spell stacking.


\begin{quote}\featname{Human Form}
\begin{small}
\shortability{Transmutation}{}
\shortability{Level:}{Brd 1; Sor/Wiz 2}
\shortability{Components:}{V, S}
\shortability{Casting Time:}{1 Standard Action}
\shortability{Range:}{Touch}
\shortability{Target:}{One Willing Creature}
\shortability{Duration:}{10 minutes/level}
\shortability{Saving Throw:}{Fortitude Negates (Harmless)}
\shortability{Spell Resistance:}{Yes}
\end{small}
\quot{The man looked at the fallen prince and smiled. He whispered some eldritch words, and then there were two princes. One living, and one dead. The living prince smiled.}

The target assumes the appearance of a specific individual of medium size or smaller, or of a generic member of a humanoid race. The target is effectively disguised, and gains a +10 bonus on Disguise checks made to impersonate the genuine article. The target suffers no penalties to Disguise for assuming the visage of a different race or sex.
\end{quote}


\begin{quote}
\featname{Lycanthropy}
\begin{small}
\shortability{Transmutation}{}
\shortability{Level:}{Sor/Wiz 3}
\shortability{Components:}{V, S}
\shortability{Casting Time:}{1 Standard Action}
\shortability{Range:}{Touch}
\shortability{Target:}{One Willing Creature}
\shortability{Duration:}{10 minutes/level}
\shortability{Saving Throw:}{Fortitude Negates (Harmless)}
\shortability{Spell Resistance:}{Yes}
\end{small}
\quot{The shaman howled in rage and transformed into a wolverine.}

The target assumes the appearance of a specific or generic animal or magical beast of small, medium, or large size. The target is effectively disguised, and gains a +10 bonus on Disguise checks made to impersonate the genuine article. The target suffers no penalties to Disguise for assuming the visage of a different race or sex. The new form is unable to use normal equipment (all carried or worn items meld into the new form when the spell takes effect), and has whatever natural weapons the caster desires (to a maximum of 1 natural weapon per four levels). These natural weapons inflict an amount of damage appropriate for a magical beast of the new form's size. Any equipment the character had is subsumed into their new form.
\listone
    \item Small, Flying:90' flight speed (good), +4 Dex, -4 strength
    \item Small, Land:+2 Dex
    \item Small, Swimming:60' swim speed
    \item Medium, Flying:60' flight speed (good), +2 Dex
    \item Medium, Land:40' land speed, +2 Strength, +2 Natural Armor
    \item Medium, Swimming: 60' swim speed, +2 Strength, +2 Natural Armor
    \item Large, Flying: 90' flight speed (average), +2 Dex, +4 strength, +1 Natural Armor
    \item Large, Land: +6 Strength, +5 Natural Armor
    \item Large, Swimming: 60' swim speed, +6 Strength, +4 Natural Armor
\end{list}
\end{quote}


\begin{quote}
\featname{Monstrous Form}
\begin{small}
\shortability{Transmutation}{}
\shortability{Level:}{Sor/Wiz 4}
\shortability{Components:}{V, S}
\shortability{Casting Time:}{1 Standard Action}
\shortability{Range:}{Touch}
\shortability{Target:}{One Willing Creature}
\shortability{Duration:}{10 minutes/level (D)}
\shortability{Saving Throw:}{Fortitude Negates (Harmless)}
\shortability{Spell Resistance:}{Yes}
\end{small}
\quot{With a sweep of your cloak you become a creature of nightmare.}

The target assumes a horrific and monstrous countenance of a monster of Medium, Large, or Huge Size. The basic structure can look like pretty much anything, and the descriptions are just guidelines. All of the character's equipment melds into his new form. The character no longer has the ability to use equipment, but has a number of natural weapons appropriate to the new form:

\listone
    \item Yeth Hound (Medium):50' speed, +4 Str, +4 Dex, Bite, Improved Trip
    \item Displacer Beast (Large): +8 Str, +2 Dex, +5 Natural Armor, 1 Primary Bite and 2 secondary Tentacle Whips, Concealment.
    \item Monstrous Spider (Large): 30' Climb Speed, +8 Str, +8 Natural Armor Bonus, 1 natural weapon Bite, Poison (1d6 Con/ 1d6 Con)
    \item Chuul (Large): 60' Swim Speed, +8 Str, +6 Natural Armor, 2 Primary Pinchers, character gains the [Aquatic] Subtype.
    \item Bulette (Large): 20' Burrow Speed, +8 Str, +10 Natural Armor
    \item Manticore (Large): 60' Fly Speed (Average) +8 Str, +6 Natural Armor, 2 natural weapon Claws, 2 natural weapon ranged spikes attacks (1d8 + Str, 19-20 crit, 20' range increment)
    \item Giant Serpent (Huge): +14 Str, +10 Natural Armor Bonus, 1 natural weapon bite, Poison (1d6 Dex damage/1d6 Dex damage), Improved Grab.
\end{list}
\end{quote}


\begin{quote}\featname{Fiend Form}
\begin{small}
\shortability{Transmutation}{}
\shortability{Level:}{Sor/Wiz 5}
\shortability{Components:}{V, S}
\shortability{Casting Time:}{1 Standard Action}
\shortability{Range:}{Touch}
\shortability{Target:}{One Willing Creature}
\shortability{Duration:}{10 minutes/level (D)}
\shortability{Saving Throw:}{Fortitude Negates (Harmless)}
\shortability{Spell Resistance:}{Yes}
\end{small}
\quot{With a foul guttural utterance and a rude gesture, the wizard transforms into a fiend from the lower planes.}

The target assumes the appearance of a specific individual of medium size or smaller, or of a generic member of a fiendish race. The target is effectively disguised, and gains a +10 bonus on Disguise checks made to impersonate the genuine article. The target suffers no penalties to Disguise for assuming the visage of a different race or sex. While in Fiendish form, the target gains two bonus [Fiend] feats of your choice that it would meet the requirements for if it was actually a member of a fiendish race, and gains access to a sphere of your choice. In order to use a spell-like ability from the sphere, the target must expend one spell-slot or prepared spell of an equal or greater spell-level, but there is no other limit to how many times the spell-like abilities can be used. Rules for [Fiend] feats and spheres may be found in the Tome of Fiends.
\end{quote}


\begin{quote}\featname{Dragon Form}
\begin{small}
\shortability{Transmutation}{}
\shortability{Level:}{Sor/Wiz 6}
\shortability{Components:}{V, S}
\shortability{Casting Time:}{1 Standard Action}
\shortability{Range:}{Touch}
\shortability{Target:}{One Willing Creature}
\shortability{Duration:}{10 minutes/level (D)}
\shortability{Saving Throw:}{Fortitude Negates (Harmless)}
\shortability{Spell Resistance:}{Yes}
\end{small}
\quot{The final incantations are completed and you transform into a dragon.}

The target character assumes the form of a huge dragon. The character gets a +14 Strength bonus and a -4 Dexterity penalty. The character gains a +18 Natural Armor Bonus. The character gains immunity to one energy type (which must be Acid, Cold, Electricity, Fire, or Poison), and a breath weapon that inflicts 1d6 per level of the same type of damage. Using the breath weapon is a supernatural ability that requires a standard action and may only be used at most once every 1d4+1 rounds. The character has a flight speed of 120' with poor maneuverability. A character in Dragon Form has three natural attacks: a primary Bite and two secondary claws. Worn equipment is subsumed into the new draconic form.
\end{quote}

%\end{small}
%\end{multicols}

\subsection{Some Effects Don't work}

\subsubsection{Stacking Spell Resistance}

Spell Resistance does stack, but it does so in a really weird way that the authors have never actually taken the time to explain. SR is a DC for a level check, and that means that it is actually calculated as the number people are supposed to roll to penetrate your SR plus your CR. A SR of 15 on a CR 1 monster is awesome (it means that people of your level are supposed to roll a 14 to penetrate your SR), while a SR of 15 for a CR 13 monster is a joke (it means that enemies fail to penetrate your SR only on a natural 1). When you have Spell Resistance, and your CR goes up, your Spell Resistance also goes up. An Imp with a CR of 2 and a SR of 6 who takes enough levels of Wizard to gain a CR (2 levels as it happens) gains 1 SR as well and is SR 7.

But what happens if more than one source gives you SR? Well, it still stacks, it just does so in the aforementioned really weird way. First, you take your highest SR, then you start adding very small numbers to it based on what your other sources of SR would give you. If a secondary source of SR is less than 6 + your CR, having it increases your SR by +1. If a secondary source of SR is 6 + your CR or more, but less than 11 + our CR, your primary SR increases by +2. If a secondary source of SR is between 11 + CR and 15 + CR, it increases the primary SR by +3. And finally, a secondary source of SR that is 16 + CR or more adds +4 to the primary SR.

It would be nice if the basic rules ever explained that, but they don't. It doesn't come up all that often, but Drow Monks, for example, don't end up with SR in the high 30s at mid level. Their SR is actually just moderately impressive.

\subsubsection{Hiding in 3.5 D\&D is Dumb}

OK, we all know that it makes us feel kind of bad when the Rogue sneaks up on people and stabs them in the face without them ever seeing who did it. But you know what? People totally do that crap all the time. It's not even an uncommon occurrence, and there's really no cause to get excited about. The 3.5 rules for hiding, where you need cover or concealment to hide, are retarded. That makes Rogues run around with tower shields so that they can hide themselves and their equipment behind the cover of the tower shield (including the tower shield itself, which makes my brain hurt). Yes, you can totally hide when there are no intervening objects between you and the victim. It's called ``sneaking up behind people" and in a game with no facing it's handled with a hide check opposed by spot.\\

If you attempt to hide in a combat setting, you are under a number of restrictions:
\listone
    \item A character who has been attacked automatically can guess what square you are in. You may retain your invisibility, but that's just Full Concealment, and they could very plausibly hit you.
    \item There is a -20 penalty to Hide for attempting to fight while hidden. The distance penalties on Spot are pretty amazing, but most people can't hide at a -20 penalty.
    \item Once they see you, they see you. If an opponent successfully spots you even once (and they get to try every round while in combat), they just plain see you until you manage to get all the way out of their field of view (generally requiring you to leave the scene or make bluff checks or something).
    \item Spot Bonuses can get quite large. A spotter who knows what he's looking for gets a +4 bonus, and a spotter who is extremely familiar with the target gets a +10 bonus -- these bonuses are weirdly listed under the Disguise skill, but they still apply (so if someone says ``There's a halfling Ninja over there!" every other Guard gets a +4 bonus).
\end{list}

\vspace*{8pt}

But you can do it. Hiding in combat is hard, but it's a thing that powerful characters may be able to do against some opponents. Some of the D\&D authors have an outdated idea that Rogues should be forced to ``hide in shadows" or something. But this is D\&D, and most enemies have Darkvision. There are no shadows. Attempting to force Rogues to hide only in areas that they could plausibly hide in if a suspicious person was looking right at them and knew what they were looking for is incredibly cruel. In any kind of stressful situation that isn't an accurate picture of what is going on.

\chapter{New Rules}

The interaction of Undead with the rest of the rules is often less than satisfactory. Part of this is that the undead type itself is extremely overzealous in the game effects it provides. The fact that all undead don't need sleep means that vampires don't have to sleep in their coffins. The fact that undead don't have a Constitution score means that Ghouls can run for exactly zero rounds before they have to make a Con check (that they automatically fail) to continue (and also says it can ``run on indefinitely", a base contradiction that makes us sad). The fact that undead are immune to critical hits means that a vampire can't be staked through the heart (even if it was sleeping, which it isn't). But even beyond that fundamental error, the multitude of authors that compromise the Dungeons and Dragons design staff never seemed to get on the same page as to exactly what being undead means -- so a surprisingly large number of contradictory statements pepper the products.

And I'm not just talking about how they made an entire Deathless Type when there's already Ghosts (Alignment: Any) right in the core rules.

\input{undead}
\input{necromanticlocations}
\section{Necromantic Equipment}

\subsection{New Materials}

\ability{Boneblades:}{Boneblades are alchemically and necromantically hardened blades made from the bones of intelligent creatures, and the material can only be created by craftsmen with the Boneblade Master feat. For an unknown reason, they only retain their special properties if they are made into light slashing or piercing weapons.  Boneblades used in melee combat ignore the damage reduction of any undead creature and can hit incorporeal creatures as if they were magic weapons with the ghost touch property.

Boneblades made from dragon bones can be combined with the Dragoncrafter feat to produce items with both properties.

Cost: 1,000 gp per lb.}

\ability{Blood steel:}{Blood steel is steel that has been mixed with the blood of certain powerful creatures, making it redder than normal steel and with unusual properties. Weapons made of blood steel do 2 additional points of damage on a successful hit and any armor made of blood steel has an armor bonus two higher.  These effects only manifest if the blood steel is allowed to hungrily latch onto its user's flesh, at which point it reduces the bearer�s Constitution by two points.  Treat blood steel as normal steel for creatures with no Constitution or creatures protected from this effect.

Cost: 2,000 gp for a weapon, 1,000 gp for armor or shield.}

\ability{Black steel:}{Black steel is steel that has been mixed with necromantically charged obsidian, making it as sharp as adamantine and as dangerous as obsidian. Weapons made of black steel count as adamantine for all effects, but perform as if enhanced with the Ghost Touch and Wounding properties (without additional cost). Characters using items made of Black steel suffer one point of Wisdom drain for every day they are held, worn or carried.

Cost: 15,000 gp for a weapon, 24,000 gp for light armor, 28,000 gp for medium armor, 32,000 gp for heavy armor.}

\ability{Tainted obsidian:}{While normal obsidian often has interesting necromantic properties, tainted obsidian is an actual source of necromantic energy.  When a magic item has is made of tainted obsidian, it can provide uses or charges of its Necromancy effects in exchange for doing wisdom drain.

Cost: +10,000.}

\chapter{Monsters}

Perhaps the most important thing about necromancy is the undead creatures themselves.

\chapter{Created Monsters: Forged and Bred}
\vspace*{-8pt}
\quot{``It's alive! Or at least animate\ldots\ it's not an object anymore, that's my point.''}

There are three entire types of creatures in D\&D that are to one degree or another created. The obvious one of course is the Construct. It's a creature which was never alive and created by sorcery. Well, most of them are like that. The Flesh Golem is kind of hard to explain actually, but whatever. The point is that every Construct is \emph{constructed}. That's the whole point. And of course there are a lot of Undead that are pretty much the same thing except that they are animated with Negative Energy channeled into them. There's a lot you can say about those guys, and we actually \emph{did} that in the Tome of Necromancy. So we aren't touching that one here. The other one of course, is the Vermin. In D\&D land, ``Vermin'' doesn't mean anything vaguely approaching its meaning in natural English. Rats and cockroaches are vermin because they live in your pantry and poop on your food stores, but they aren't \emph{Vermin} because they aren't enormous biological constructs that mindlessly follow the programming planted in them by an ancient race of long departed mage kings. Really. That's what the Vermin type means in D\&D land. Actual giant insects are just animals in the same way that dire toads and weasels are animals.

\section{Vermin: Remnants of a Fallen Empire}
\vspace*{-8pt}
\quot{``Great holes secret are digged where earth's pores ought to suffice. Things have learnt to walk which ought to crawl\ldots''}

\desc{Ants track by smell and follow trails left by other ants and bees see deep into the ultraviolet spectrum and perceive a beautiful tapestry of gorgeous colors that escape the eye of the man and the mouse. And when dealing with Vermin type creatures that all means precisely \emph{nothing}, because Vermin in D\&D don't do any of that. It's not because the scent ability was ``left off'' the Monstrous Ant description, it's because the Ant described in the Monster Manual genuinely doesn't have a good sense of smell. It does have Darkvision out to 60 feet like an outsider or a Construct, and that's not an accident either despite the fact that Earthly ants really demonstrably don't do that regardless of size.}

The Monstrous Scorpion isn't a super sized scorpion \emph{at all}. It has a set of abilities which are on the face of it completely bizarre from the context of what actual scorpions do, because it's actually a living construct created by a long fallen empire for use in war. That's why it's immune to hallucinatory poisons and can see in perfect darkness. It's actually created from biomass by powerful magic and not by the interaction of natural and magical mutation across a thousand generations and a harsh selection process hastened by unpredictable climate and predation by manticores. The Vermin have a couple of neat things going for them which is why they were created as war machines in the first place:

\listone
	\bolditem{Mindless -}{Unlike actual or even giant spiders, the \emph{monstrous} spider has no mind at all. It cannot be influenced with magic or confused with poisons. It can't even be detected with \spell{detect thoughts}.}
	\bolditem{Brainless -}{Vermin are subject to critical hits because they have segments and organs, but they don't have any \emph{brains}. That means that they can be blinded, but not killed, by decapitation.}
	\bolditem{Darkvision -}{Vermin can see even in complete darkness, making them quite useful in cave fighting.}
	\bolditem{Aggressive -}{The vast majority of predators will retreat from battles where they are presented with even a chance at serious injury. Yet Vermin fight until they are dead. That's a really bad plan for an individual or even a species, but it's \emph{great} for a battle platform.}
\end{list}

\subsection{Who made the Vermin?}
\vspace*{-8pt}
\quot{``They did not \emph{know} that steal marks flesh, and they did not \emph{know} that flesh does not mark steal. In their ignorance they continued to do one task after another in the old ways. They did not \emph{know} what we \emph{know}.''}

Vermin come from the before time. The time when metals were not made and words were not written down. It's quite a feat of construction talent and a testimony to the power and ingenuity of these ancient flesh crafters that these devices are still running, still attempting to fulfill their programming to this day. The answer is not known in the days that D\&D is normally set. They are a product of a bygone age and their origin is a mystery to all but the Aboleth and the memory fish are being \emph{extremely quiet} on this subject. And yet, their conspicuous silence is probably more telling than anything they could possibly say. The Vermin were constructed during the days when aberrations ruled the world, and they were quite obviously designed to fight against aberrations.

\abox{Getting the Program}{All Vermin have a program that they follow at all times, usually involving a spiral search pattern in groups of one to six until they encounter a creature, at which point they will attack it until it is dead. If confronted with more than one type of creature, they will target them in the following order:
\listone
	\item Any Aberration (except their specific non-targeted group)
	\item Any Humanoid
	\item Any other moving creature they can detect
	\item Anything especially edible
\end{list}\vspace{2pt}

This behavior is entirely comprehensible from the standpoint of the wars in the before time -- the Ilithid and Aboleth both sent slave troopers to their death by the millions in their quest for world domination.}

\desc{Individual groups of Vermin will usually have one type of aberration that they will not ever attack. It may be an entire race of aberrations (such as Kopru or Neogi), or it may be a specific clan of aberrations (such as the Aboleth spawn of the Great Mother of the Howling Wells, or the Ilithid of the Tallow Halls). In any case, determining the type of Aberration that is completely safe from any group of Vermin can be done by observing the markings on the beast. Extracting that information is a DC 30 Knowledge Nature check.}

Vermin eggs persist apparently indefinitely and are produced by the hundred score. A starving Vermin cocoons itself and goes into a state of hibernation so deep that it is essentially mummified. When in the presence of magical auras, the eggs of Vermin progress steadily towards hatching, and the cocoons burst forth their contents. Thus it is not weird or unexplainable for areas that recently have been subjected to incursion by adventurers or mind flayers to spontaneously develop invasions by tiny monstrous centipedes or giant cocooned spiders.

\subsection{Vermin Alchemy}
\vspace*{-8pt}
\quot{``The old ways are the good ways.''}

Vermin cannot think for themselves, nor would they have been better at their job given that ability. So it is not surprising that one can severely adjust the behavior of Vermin through the use of chemicals and sounds. Identifying the sounds and smells that a particular group of Vermin will respond to is difficult (requiring a DC 30 Knowledge (Nature) check), but actually producing them is not particularly. Here is a list of possible behavior modifications one can achieve and the Perform or Craft (Alchemy) check required:

\listone
	\bolditem{DC10 - Rampage}{It's a very simple behavior modification to cause a rampage. The spiral search pattern ends entirely and all affected Vermin take off in a random direction and move at full speed or until their path is blocked by a creature.}
	\bolditem{DC15 - Ignore}{There are chemicals that cause Vermin creatures to simply ignore}
	\bolditem{DC20 - Attack}{}
	\bolditem{DC20 - Shut Down}{}
	\bolditem{DC35 - Command}{}
\end{list}

\section{Constructs: Durability at a Price}

\desc{Like the Undead, the Constructs suffer tremendously from the fact that they have been over generalized. It is of course thematically appropriate for a Golem to be tireless and work day and night on whatever its last command was for as long as day follows night and night follows day. But it is also thematically appropriate for a clockwork beast to wind down and ``pass out'' as it continues to work long or strenuous schedules. Similarly, while it is fine and more than fine for an implacable lump of animated steel to be immune to critical hits, the very idea that there aren't key locations on a geared robot or a colossus given life by a mystical forehead rune is patently ridiculous. The construct type, therefore is filled to the brim with stuff that has no business being there, and this harms the game. The immersiveness of the story is depleted when players cannot rationally deduce what effects a being is resistant or vulnerable to, and anyone who's ever slapped washing machine or tripped over a playstation knows that there's no excuse for a machine to be immune to stunning.}

So here it is, the Construct Type. Pared down to the things it should actually do. Remembering of course that the Type itself should contain only those effects that one would want to be a universal law for all constructed beings, rather than rules one could imagine being situationally appropriate for one construct or another:

\listone
	\bolditem{Low Light Vision:}{Sees twice as far in limited illumination.}
	\bolditem{Dark Vision:}{60'}
	\bolditem{Poor Healing:}{Constructs can be healed by any of a number of means but do not heal for periods of rest. A construct's daily healing rate is 0 hp (though of course a construct with Fast Healing has a healing rate \emph{per round} and likely doesn't care).}
	\bolditem{Mindless:}{Even an intelligent construct has a synthetic mind that is unreachable by sorcery. A construct is not affected by [Mind Affecting] effects and cannot be detected with \spell{detect thoughts}.}
	\bolditem{Never Alive:}{A construct cannot be raised or resurrected. A construct is likewise immune to energy drain.}
	\bolditem{Repairable:}{A construct does not become staggered at 0 hit points, nor does it die at -10. If for some reason you are using the ``Death by Massive Damage Rule'', constructs aren't affected by it. As soon as a Construct hits zero hit points it becomes inert, and any abilities it may have cease to function (including fast healing abilities). However, a construct in this state can still be brought to working order again with a Craft check with a DC equal to the DC to make it in the first place with a base amount of time of one hour per hit point below 1 the construct was left at.}
	\bolditem{Nonbiological:}{Constructs do not eat or breathe, constructs do not age.}
	\bolditem{Lacks Squishy Bits:}{A construct is not affected by any effect that allows a Fort save unless that effect affects objects or is a (Harmless) effect. For example, a clockwork horror is not going to catch red fever or become nauseated by a stinking cloud. But it is not outside the realm of possibility for an eidolon to be afflicted with a totally magical disease that functions off of Willpower saves.}
\end{list}

All the stuff about constructs being ``immune to necromancy'' is out the window (because we all know that you can use \spell{magic jar} to put your soul into a statue); all the stuff about constructs being immune to ability damage is out the window (because we all know that you can slow down a lumber construct); and of course the immunity to critical hits is \emph{totally} out the window (if you have the name of Pelor on your forehead there is at least one critical location that probably won't go well for you if it is hit).

\subsection{Controlling Constructs: Robot Armies and Statuary Servants}

\desc{Time and time again adventurers report finding constructs that have been left attending temples and castles long after those buildings have fallen into ruin. The reason for this is twofold: First, constructs don't age; and Second, constructs don't count as one of your eight constant magical items if they are set to guarding a location. This means that powerful wizards are actually encouraged to leave their golems places with patrol or sentry orders and then of course these sentry golems will have a tendency to outlive the wizards, and even the buildings that they guard.}

Of course, it's entirely possible to make your constructs follow you around. If you do, they count against your 8 item limit.

\abox{Behind the Curtain: Why the Lower CRs?}{A cohort, or a planarly bound outsider, or a necromantically crafted monster could all plausibly be of a CR that is just 2 less than your character level without particularly disrupting play. So it may seem pretty weird that the constructs one can order around are weaker than that. The reasoning is ironically because the tactical role of a construct is so different from that of a Ghoul or Jarilith. While many potential servant creatures are simply weaker versions of normal characters or dangerous and fragile glass cannons -- in almost all cases a construct is an offensively anemic unit with a highly powerful defense. For those of you who have played tactical games or MMOs, that makes the average construct an ideal ``pet''. A strong defense is disproportionately useful for secondary characters expected to travel in front, and the fact that characters aren't allowed to fill their magic item cap with cohort level constructs is no accident.}

\subsection{Specific Constructs Under the New Rules}

\input{monsters/simulacrum}

\subsection{Other Constructed Entities}

\subsubsection{Robot Girl}
\quot{``Chiu...''}

\desc{While often constructed beings are nonliving in the strictest sense, sometimes a spellcaster or inventor will take it upon themself to build a being which is genuinely a living creature, albeit one which is composed of ceramic and metal rather than flesh and bone. Sometimes this is to create a servitor being which is less imposing than golems but more stable than simulacra, sometimes it's for other purposes.}

\desc{\listone
\item Type: Humanoid (Construct)
\item Ability Scores: no modifications
\item Medium: As Medium creatures, Robot Girls have no special bonuses or penalties due to their size.
\item Robot Girl base land speed is 30 ft.
\item Skills: +4 racial bonus to Search
\item Alignment: anything, really
\item Ghost in the Shell: As a creature with the Construct subtype, the Robot Girl has a Con score and everything, lacking Construct traits, but is treated as a Construct by any effect that specifically works differently against them, such as a Mace of Smiting or Favored Enemy (Constructs). Likewise, they can choose to count as a Construct for the purpose of taking feats, prestige classes and the like. Also, they have souls, and are partially made of metal, gaining a +1 Natural Armour Bonus.
\item Built: The Robot Girl can be healed by Repair spells as well as Cure spells, and gains a +4 bonus against Poison and Disease. They need not sleep, but can do so if they want. They still need food.
\item Lightning-Powered: They gain Electricity Resistance 10 + Hit Dice. Any time any Electricity is resisted, they count as having eaten a full meal. It is delicious.
\item Laser Eyes (Su): Can be used once per two rounds, with a 30' range. Requires a Ranged Touch Attack, and deals 1d4 Light damage per hit die.
\item Scanner (Sp): You may cast Detect (Magic/Poison/Evil/Good/Chaos/Law) at will.
\end{list}}

\descfeat{Steel Angel}{You are extra robotic but extra sweet}
\textbf{Requirement:}{Robot Girl}

\ability{Benefit:}{You gain Damage Reduction/Adamantine equal to your hit dice, and any [Charm] effects you use gain a +1 Racial bonus to the save DC.}

\ability{5 HD:}{Once per day you can fire a salvo of missiles from your shoulders/wrists/knees/boobs as a (Su) ability, out to 150'. They explode in a 30' burst, dealing 1d6 Fire damage per HD and 1d6 Sonic damage per HD, Ref half (Con-based).}

\ability{10 HD:}{You gain a Fly speed of 30' (Good) as you jet around the place.}

\ability{15 HD:}{Once per day you may fire a Photon Beam from your face/chest as a (Su) ability. This extends in a 100' long line, hitting everyone in the area with a Disintegrate effect. The save DC is Con-based.}

\section{Denizens of the Planes of Law}

When you think avatars of Evil in D\&D it is no trouble at all to conjure up images of spiteful devils and destructive demons; but when you talk about a being of \emph{Law} the image that comes up is simply not the same from one person to another. Part of that is because Law doesn't really mean anything consistent in D\&D nomenclature. And part of that is because the actual description of the inhabitants of Mechanus has changed wildly through the generations and editions.

\subsection{Modrons: Singularity of Purpose}
% By Frank Trollman

\desc{For those of you who don't remember: Modrons are the original creatures of Law from the old days of AD\&D. They haven't been seen very often because they were originally written as a joke. Their very existence is as offensive to many players as the fact that they were essentially retconned out of existence is to others. And what's that all about? It's because the Modrons were originally written up as giant dice. Yes, really. The different types of basic Modron are shaped like four sided dice, six sided dice, 8 sided dice, the whole thing.}

\desc{So if your DM jumps on the ``let's forget this ever happened'' bandwagon, we understand. The original write up of the Modrons was actually pretty insulting. But since then there have been a number of variously successful attempts to rehabilitate them and make them independently awesome. Different Modron art has been made by Tony DiTerlizzi and Eric Campanella that looks pretty darned awesome � and not like your DM put a 6 sided die on the battle mat at all. Instead each Modron looks like a ghastly hybrid of metal and flesh covered with cogs and wheels where spindly appendages emerge from a solid (though not rollable) core.}

\desc{So assuming that you use some of the reform Modrons from late in 2nd Edition, the Modrons are actually pretty cool. They represent the idea of Law as an implacable and incomprehensible force. They are at their best when portrayed as being so single mindedly focused on some long term goal that they actually don't even care about you. Sometimes they destroy your village, sometimes they don't, and there's really no predicting that sort of thing unless you're knowledgeable about the Big Plan. Now I know what you're thinking� that having a plan so convoluted and far ranging that mortal minds cannot grasp it or predict its unfolding is actually indistinguishable from not having a plan at all and just performing actions at random. And yeah� that's true. That's DnD alignment for you.}

\desc{The Modrons come from a city in the Clockwork Nirvana called Regulus and have a rigid caste system where more powerful Modrons are told more of ``the plan'' than less powerful Modrons are told less. Each Modron is told exactly as much as it needs to know to complete its assigned tasks. And in the face of a long term plan of this magnitude, that pretty much means that every Modron is kept entirely in the dark about just what the heck it is doing or why it is doing it. The Modrons are arranged into a rigid caste system with no possibility (or concept) of personal advancement. That would be pretty stultifying if they were like humans where they all started out equal, but they aren't like that at all and Modrons of each caste sincerely don't have any desire to move up to another caste. At the very bottom (or at least, ``most numerous and least clued in to the plan'') there are the cogs. Cogs look like little gears and have the ability to transform into any object of roughly a cubic foot or less. When properly supervised by a higher level drone, they can take the shape of quite complicated pieces of clockwork and frequently do so. Above them are various drones of various shapes and sizes, each constructed of materials mechanical and biological to fulfill its role in the great plan.}

\desc{At the top there is Primus, who has been variously described as everything from an Intelligent Item to a reasonably powerful Outsider to a guy working the machinery behind a curtain. Seriously, your campaign can potentially reveal anything you want it to about what is really in the center of Regulus because that's been retconned so many times that noone knows what the official answer even is right now. Primus has been killed off several times in official continuity as part of various authors attempting to delete the Modron race from D\&D. However, someone always brings Primus back, because D\&D never really throws anything away (except the pygmies, that was too racist even for the 1980s). If you demand continuity in your life, then it seems that Primus simply can be killed time and time again, each time getting replaced with a new Primus who is in turn enlightened as to the nature of the big plan and granted the authority to turn the wheels of Regulus.}

\desc{The Modrons mostly sit around and operate the machinery in Regulus that apparently keeps all the giant gears and pistons running in the Clockwork Nirvana of Mechanus. So even if they exist in your game's continuity, chances are good that you'll never encounter them. Every so often, a whole lot of Modrons open up gates to other planes and start wandering around in big groups doing� stuff. The amount of time that Modrons spend between their ``March'' is supposed to be constant but actually every single Modron March that has ever been mentioned in any published adventure or story has taken place out of sequence so one is forced to conclude that actually the Modron March happens whenever the big plan calls for it and rumors of a great schedule are just rumors. The Modron March is generally not harmful and the Modrons don't seem to go out of their way to chase anyone who gets out of the way, so it doesn't seem to be an invasion. Although who knows? The plan of the Modron is so incredibly far reaching that it is entirely possible that they simply walk around in huge armed groups wandering seemingly aimlessly through the planes at irregular intervals doing no harm to anyone so that at some later date they can do the same thing and then just destroy some enemy that is predicted in the distant future.}

\desc{Unfortunately, now I'm going to have to talk about the Hierarchs. These are a layer of bad asses that live in the eco niche between drone and god. They are roughly equivalent to the high end fiends and celestials � coming in various flavors and power levels like Mariliths and Balors do for the demons. Unfortunately, noone has ever overhauled the art on these guys to the point where they don't look like ass. Sorry, the Hierarchs of Regulus look like they were drawn by Napoleon Dynamite and there's nothing nice I can say about them. If I were personally inclined to run with these guys at all I would be forced at gun point by the players to have them look like� something else. Putting them back into the theme of the Modron Drones is probably best, because at least then they appear to be Modrons in the same way that a Trumpet Archon is readily identifiable as an Archon. Which means that honestly the Hierarchs are much cooler if they look like Daleks. And the truth of that statement is probably the single greatest argument for walking away from the whole thing, as advised by Monte Cook.}

\desc{So what do Modrons do in a story? Mostly they show up with a specific set of instructions that they attempt to fulfill. They speak their own language but the more powerful ones also speak additional planar languages (hope you speak Formian). If the player characters attempt to prevent the Modrons from doing their thing, the Modrons will fight. Otherwise they'll simply complete their task and leave. If other creatures attempt to stop the Modrons, they'll fight them. Modrons hold no grudges and have no loyalty but to the plan. They will seamlessly switch sides in a battle if a different group proves more detrimental to their mission. The Modrons don't know why they are doing what they are doing, only Primus does (assuming Primus exists, in some versions The Plan is actually a flaw in Mechanus and there is no reason for any of it). And yet they will fight to the death to complete their mission. With good art and weird dialogue, the Modrons can serve as dynamic antagonists or useful allies.}

\desc{Like an amazingly intricate puzzle box, the Modron Race unfolds, collapses, and progresses in countless ways, but at the end, it's still a puzzle box. The entire point of the Modrons is that you never get clued in to what they are doing or why they are doing it. But if you don't want to use them at all, I understand.}

\subsection{Inevitables: Enforcing Natural Law}
%By Quantumboost

The Inevitables are unusual among ``exemplary'' creatures from Mechanus in that they actually don't have the Outsider (Law) type; they're straight-up Constructs. They're still Extraplanar, and they still have the Law subtype, but they're forged from steel and other metals, or at least the closest Outer Planar equivalents. What they represent among the creatures of Law is the idea of Law as ``universal rules'' -- that there is some code which should be obeyed by all beings, and that if you don't abide by it super fighting robots from another dimension will come and stop you from disobeying if and when they find out.

\subsection{Formians: In Many, One}
%By Quantumboost

\quot{``There is no good for the bee that is bad for the hive.''}

\desc{The Formians are the current Lawful Outsiders, and true to their name are giant ants -- and that's most of what you need to know about them. Formians represent Law as Conformity -- conformity to their society, and removal of anyone who won't conform. They're expansionistic, militaristic, and each hive effectively acts as one organism as they spread across the gears. Like most portrayals of giant ant-people, Formians are telepathic and have a hive-mind. However, they are giant ant \emph{people}, and each individual Formian above Worker level has a level of mental ability at least what you might expect from a human. So interacting with them is a lot like working with the Borg - you can totally }

\desc{The most straightforward way to use the Formians is to have them show up and want to conquer your favorite hangout -- either taking away the people to work in their hive (which is a lot like what the Hobgoblins will do, except that there probably isn't any chance of advancement from the slave thing) or turning it into a new hive, superintelligent mind-enslaving queen ant and all. But you can also use them as more mundane antagonists or even as potential allies. Unlike Modrons, Formians actually have discernable goals and motivations, their motivations just all happen to be ``what is good for the hive''. If you're working towards a goal or offering a reward that they think will help the hive, they'll totally be willing to work with you and might not even bother you about the whole assimilation thing beyond handing out some pamphlets.}



\end{document}