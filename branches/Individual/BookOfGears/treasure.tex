\chapter{Treasure and the World}

\section{Finding Treasure}
\vspace*{-8pt}
\quot{``There's nothing here but worthless gold!''}

\desc{It is an absolutely necessary step in the entire process of dragon slaying that one cart off the pile of gold. Indeed, previous editions oft as not required that one employ literal carts to carry off the fantastic wealth that a single Dragon might hoard. This was made possible by the letters G and H and by the number nine \emph{thousand}. And while it is true that the old alphabetical treasure types may have been a \emph{bit} overboard with the tremendous piles of loot that they handed out, the reverse trend of giving characters piles of gold that fit in one's pocket is fairly unfortunate.}

It is absolutely the case that any dragon worth its salt should be worth enough in gold that it can actually \emph{sleep} on said gold. For reference, that's about 760 pounds of gold for a minimal medium-sized dragon, and about three tonnes of gold for a large dragon. But it is equally the case that when you encounter a group of gnolls or bugbears or even hill giants they generally don't have a big pile of gold and more often than not they don't have any magical items. Even more importantly, owlbears don't have any treasure \emph{at all}. Their digestive systems really will destroy all of the valuables they eat, and most of the time they won't even eat valuables because \emph{they're owlbears}! They live in the woods and they kill things and they don't participate in any economic activity at all.

\subsection{Books}

\desc{One of the most important and interesting things one can find in a cooperative storytelling game is a book. It's a story within a story, a source of potentially needed information and it's not really game breaking for your character to have it. It is for this very reason that virtually every Dungeon Magazine includes at least one book that the characters can find.}

So what do books actually do? Well, the obvious thing is that if there are any spells in them, you can copy those down into your spellbook (or your \spell{secret page} manifest pad, if you're a modern wizard). But even if they are completely mundane they can still be useful. If you have enough of them on a subject you can have a \emph{Library}, which allows you to take 20 on Knowledge Checks. And a book about a specific subject can allow a character to spend an hour in study to make a knowledge check as if you had an appropriate Area of Expertise. So if you are confronted with a hobgoblin wartabard, and you can't make a sufficient Knowledge Geography or Knowledge History check to figure out where it comes from -- you can bust out a copy of \underline{Bastions of the Goblin Khanate} and try to find a match -- then you can make another knowledge check with the much lower DC.\\

\textbf{One Hundred Books that you can find in a Fantasy Setting:}
\begin{enumerate}
	\topsep=0pt
	\itemsep=-2pt
	\item The Ascendancy of Fire
	\item Abjuring Minor Demons
	\item Alterationism and Revisionism
	\item August and Winter
	\item Anatomie d'Ghoule
	\item Book of the Wars of Pelor
	\item Bastions of the Goblin Khanate
	\item Bees: Keeping and Secrets
	\item Benevolence and the Duchess
	\item Blzht's Personal Notes on the Badger Kingdoms
	\item Birthrights
	\item Crumbling Shadow
	\item The Cruelty of Healing Magic
	\item Carbuncles in an Elvish Context
	\item Cyclopean Constructions of the Vanished Ones
	\item Cults of the Maggot God
	\item De Vermis Mysteris
	\item The Draconomicon
	\item The Diary of Jakkar the Mad
	\item The Diary of Prince Olaf
	\item Dangerous Plants of the Bane Mires
	\item Djinn Fermentation Techniques
	\item Donjon Menagerie
	\item 101 Secrets of Devilcraft
	\item Eternal Subjects: Stasis and Crystal
	\item Evil
	\item Etherealness, Property, and Government
	\item Extreme Cold: A Goblin's Tale
	\item Ettercap: The Terrible Secret Reality
	\item The Crawling Darkness: Practical Necromancies
	\item Fabled Lands and Mythic Locales
	\item Fairy Courts: Sun and Shadow
	\item Fear in Hoburg
	\item Remnant Cities and Constructions of the Ancients
	\item The Bestiary Arcane
	\item Giant Crab!
	\item The Giant Kingdom: A Traveler's Perspective
	\item Gargoyle Physiology
	\item Grafting Flesh and Lead
	\item Gold: Providence and Necessity
	\item Gnome Lore
	\item Horror and Birds
	\item Harpy Statecraft
	\item The Harvest of Sorrows
	\item Blood of the Innocent
	\item The Asmodeus Gambit
	\item Blood and Silk: Danger Rises as the Sun
	\item The Cutting Edge: A Warrior's Tale
	\item The Dark History of Bladereach: A shocking and true revelation
	\item Fantastic Economie
	\item The Fly and the Serpent: Against the Giant Frog
	\item Cults: Demons: Dark Miracles
	\item Five Beans You Can Eat
	\item The Broken Mask: A Practical Guide to Hunting Shapeshifters
	\item The Book of Odamma
	\item Children of a Dark Star
	\item The Horrible Reality: The Devouring Darkness Unavoidable
	\item Last of its Kind: Twelve Dying Races
	\item Dark Revelations V through IX
	\item The House of Fiery Justice
	\item Aboleth Memories
	\item Industrial Uses of Slimes, Molds and Jellies
	\item Balance and Leverage: Druidic Construction and the Natural Order
	\item The Path of Blood
	\item Plains of Dust on the Planes of Water
	\item The Properties of 120 Magical Plants available anywhere on the planes
	\item Political Maneuvers of the Depraved
	\item Playing with Fire: The Dangers of the Vilest Necromantic Arts
	\item The Planar Political Primer
	\item The Physiology of Pain
	\item Prophecies of Profleggathron the Ever Burning
	\item Ash on the Wind: the Conquest of Valdrana
	\item Stone Unyielding, Impressions and Sand
	\item Servants of Leaf and Branch: Dryads, Nixies, and Nymphes.
	\item Secrets of Life and Death
	\item Slaves to the Black Tower
	\item The Complete Dwarven Histories: Volume XVIII: The Seventh Bugbear Confrontation
	\item A Transcript of the Trial of Harakhdar the Forsaken
	\item Taxidermy for Profiteers
	\item A Treatise on the Efficacy of Fungal Remedies
	\item Unaussprechlichen Kulten
	\item Ur Priest: Eating the Gods
	\item The Void and the Flame: The Story of Elothar
	\item Tactica Implacable: A Primer for Dwarven Strategists
	\item Land Grants of the Wendish Borderlands
	\item Surprisingly Delicious Things
	\item Tracking the Wily Displacer Beast
	\item Potion Miscibility
	\item The Worst the Banemires Can Do
	\item Wanderers of the Void: Giant Frog
	\item Six Problems of Classic Philosophy
	\item Twelve Ninja Clans
	\item The Xorn and the Unicorn: Root and Stone
	\item Path of the Mud Sorcerer:
	\item The Wish Economy and the Brass Sultan
	\item Metallurgical Properties of Mithril and its Common Alloys
	\item The Precepts of Hruggek
	\item Your Word Against Mine: The Kobold Problem
	\item Anathema
	\item Zone Agents
\end{enumerate}

\section{The Three (or so) Economies}
\vspace*{-8pt}
\quot{``I'll give you five pounds of gold, the soul of Karlack the Dread King, and three onions for your boat, the Sword of the Setting Sun, and that cabbage\ldots''}

\desc{Life in D\&D land is not like life in a capitalist meritocracy with expense accounts and credit cards. There is no unified monetary system and there are no marked prices. \emph{All} transactions are essentially barter, and you can only trade things for goods and services if people genuinely believe that the things you are trading have intrinsic value and the people you are trading to actually want those specific things. Gold can be traded to people only because people in the world genuinely think that gold is intrinsically valuable and that they want to own piles of gold.}

That means that in places where people don't want gold -- such as the halfling farming collective of Feddledown, you can't buy anything with it. It's just a heavy, soft metal. But for most people in the fantasy universe, gold has a certain mystique that causes people to want it. That means that they'll trade things they don't need for gold. But no matter what they are giving up they aren't ``selling'' things because money as we understand the concept doesn't really exist. They are \emph{trading} some goods or services directly for a physical object -- an actual lump of gold. Not a unit of value equivalency, not a promise of future gold, not a state guaranty of an amount of labor and productive work -- but an actual physical object that is being literally traded. And yeah, that's totally inefficient, but that's what you get when John Locke hasn't been born yet, let alone modern economic theorists like Adam Smith, Karl Marx, or Benito Mussolini. If you really want to get into the \emph{progressive} economic theories that people are throwing around with a straight face, go ahead and check out theoreticians like Martin Luther, Thomas Aquinas, Sir Thomas Moore, or Zheng He. If you want to see what \emph{conservative} opinions look like in D\&D land, go ahead and read up on your Draconis, Li Ssu, Aristotle, or Tamerlain.

\subsection{The Turnip Economy}
\vspace*{-8pt}
\quot{``We got rats! Rats on sticks!''}

\desc{Most settlements in a D\&D setting are really small and completely unable to sustain any barter for such frivolities as gold or magical goods. The blacksmith of a hamlet does not trade his wares for silver, he trades them for \emph{food}. He does this because the people around him are farmers and they don't make enough surplus to hoard valuable metals. So if he took gold for his services, he would get something he couldn't spend, and then he wouldn't be able to eat. So even though people in the tiny villages you fly over when you get your first gryphon will freely acknowledge that your handful of silver is worth very much more than their radishes, or their tin cups, or whatever it is that they produce for the market, they still won't trade for your metal because they know that by doing so they run the risk of starving to death as rich men.}

\desc{The economy of your average gnomish village is so depressed by modern standards that even the \emph{idea} of wealth accumulation and currency is incomprehensible. But the idea of \emph{slacking off} is universal. There is a static amount of work that needs to be done on the farm each year and the peasants are perfectly willing to put you up if you do some of their chores. Seriously, they won't let you stay in their house for a copper pfennig or a silver ducat, but they \emph{will} give you food and shelter if you cleanout the pig trough. They have no use for your ``money'', but they do need the poop out of the pig pen and they don't want to do it. On the other hand, they also don't want to be eaten by a manticore, so if you publicly slay one that has been terrorizing the village the people will feed you for free pretty much as long as you live. That's why people pay money to bards. Bards spend a lot of time in cities and actually will take payment in copper and gold. And if they sing songs about you, your fame increases. And fame really is something that you can use to buy yourself food and shelter from people in the turnip economy.}

\desc{``Costs'' in the turnip economy are extremely variable. In lean times, the buying power of a carrot is relatively high and in fat times the buying power of a cabbage is very low. It is in this way that the people in tiny hamlets get so very screwed. No matter how much they produce or don't produce, they are pretty much going to get just enough nails and ladders and such to continue the operations of their farms. However, such as there is a unit of currency in the barter economy of the turnip exchange -- it's a unit of 1000 Calories. That's enough food to keep one peasant alive for one day. It's not enough to feed them well, and it's not enough to make them grow big and strong, but it's enough so that they don't actually die (for reference, a specialist eats 2000 Calories a day to stay sharp and an actual adventurer eats 5000 Calories a day to maintain fighting shape). In Rokugan, that's called a Koku, and in much of Faerun it is called a ``ration''. It works out to about 2 cups of dry rice (435 mL), or a 12 oz. steak (340 g), or 5 cups of black beans (1.133 kg), or 4.4 ounces of cooking oil (125 g).}

\desc{Higher Calorie foods like meat and oil are more valuable and lower calorie foods like celery or spinach are less valuable because a lot of people exist on the razor's edge of starvation. The really fatty cuts of meat are the most valuable of all (it's like you're in Japan or Africa in that way). The practical effect of all of this is that people who have a skilled position such as blacksmith or scribe get enough food to grow up big, healthy, and intelligent. The peasants actually are weak and stupid because they only get 1000 Calories a day -- they won't die on that but they don't grow as people. This also means that the blacksmith's son becomes the next blacksmith -- he's the guy in the village who gets enough food to get the muscles you need to actually be a blacksmith.}

When you start a party of adventurers, note the really tremendous expenditures that were required to make your characters. A 16 year old first level character didn't just get a longsword from somewhere, he's also been fed a non-starvation diet for 5844 days. That means that at some point your newly trained Fighter or Rogue seriously had someone invest thousands of Koku into him to allow him to get to that point. If your character is a street rat or a war orphan, consider where this food may have come from. Perhaps when the orcs destroyed your village leaving your character alone in the world the granary survived and your character had a huge supply of millet to sustain himself until he could hunt and kill deer to augment his diet.

\abox{A Note on Peasant Uprisings}{\desc{Peasants may seem like they get a crap deal out of life. That's because they do. And regardless of whatever happy peasant propaganda you may have seen, peasants aren't really happy with their life even under Good or Lawful rulership. That's because they work hard hours all year and get nothing to show for it. So the fact that they don't get \emph{beaten} by Good regimes or \emph{stolen from} by Lawful regimes doesn't really make them particularly rich or pleased.}

\desc{In Earth's history, peasant uprisings happened about every other generation in every single county from Europe all the way to China all the way through the entire feudal era (all 1500 years of it). It is not unreasonable to expect that feudal regions in D\&D land would have even more peasant uprisings because the visible wealth discrepancies between Rakshasa overlords and halfling dirt farmers is that much more intense. Sure, as in the real world's history these uprisings would rarely win, and even more rarely actually hold territory (if lords can agree on nothing else, it is that the peasants should not be allowed to rise up and kill the lords). The lords are all powerful adventurers, or the family and friends of powerful adventurers, so the frequent peasant revolts are usually put down with \spell{fireball}s and even \spell{cloudkill}s.}

Students of modern economic thought may notice that cutting the remote regions in on a portion of the central government's wealth in order to buy actual loyalty from the hinterlands could quite easily pay itself off in greater stability and the ability to invest in the production of the hinterlands causing the central government's coffers to swell with the enhanced overall economy and making the entire region safer and stronger in times of war ? but as noted elsewhere such talk is considered laughable even by Lawfully minded theorists in the D\&D world. After all, since abstract currency doesn't see use and the villagers don't have any \emph{gold}, it is ``well known'' that it is \emph{impossible} to make a profit on investment in the villages. The only possible choices involve taking more or less of their food as taxes/loot as that is all they produce.}

\subsection{The Gold Economy}
\vspace*{-8pt}
\quot{``What pleasures can I get for a diamond?''\\
``We'll\ldots\ have to get the book.''}

\desc{People who live in cities mostly trade in gold. This is not just because living so far away from the dirt farmers makes the hoarding of turnips as a trade commodity a dangerous undertaking -- but because people living in cities are surrounded by a lot of \emph{people} who provide a wide variety of goods and services they are willing and able to trade for substances generally acknowledged to be valuable rather than trading directly for the goods and services that they actually want. These valuable substances range from precious metals (copper, silver, gold, platinum) to gems (pearls, rubies, onyx, diamond) to spices (salt, myconid spores, hellcandy flowers). In any case, these trade goods are traded back and forth many times before they are ever used for anything.}

\desc{When someone sells an item or a service for trade goods they are doing it for one of two reasons. The first is that they want \emph{something} that the buyer doesn't have. For example, a man might want a barrel of lard or a bolt of silk -- but they'll accept silver coins or something else that they are reasonably certain they can trade to a third party for whatever it is that they are actually interested in. Whoever is using the trade goods is at a disadvantage in the bargaining therefore, because while they are getting something they actually want, the other trader is essentially getting the \emph{potential} to purchase something they want once they walk around and find someone who will take the silver for their goods. It is for this reason that the purchasing power of gold is shockingly low in rural areas: a prospective trader would have to walk for days to get to another place he might actually spend a gold coin -- so all negotiation essentially starts with buying several days of the man's labor and attention. The second reason for accepting a trade good is the belief that the trade good may itself become more valuable. Indeed, when were crocodiles take over a nearby village all the silver becomes a lot more interesting. This sort of speculation happens all the time and is incredibly bad for the economy. People and dragons take enormous amounts of currency out of circulation and the resulting economic downturns are part of what makes the dark ages so\ldots\ \emph{dark}.}

Gold and jewels \emph{can} be used to purchase magic items that aren't amazingly impressive. No wizard is ever going to make a masterpiece just to sell it for slips of silver. However, there are more than a few magicians who would be willing to invest some time in order to get a handful of gold that they can use to live their lives easier with. Making even Minor magic items is hard work, and wizards demand piles of gold to be heaped on them for producing even magical trinkets. And because these demands actually work, there's really no chance to purchase anything that would take a Magician a long time to make. That means that Major magic items cannot be purchased with standard trade goods \emph{at all}. There's literally no artificer anywhere who is going to sit down and make a Ring of Spellstoring or a Helm of Brilliance in order to sell it for gold -- because the same artificer can acquire as much gold as he can carry just by making Rings of \spell{Featherfall} or \magicitem{Cloaks of Resistance}.

\subsection{The Wish Economy}
\vspace*{-8pt}
\quot{``They scour the land searching for relics of the age of legends. Scant remnants they believe will grant them the powers of the Vanished Ones. I do not. The Age of Legends lives in me.''}

\desc{Magicians can only produce a relatively small number of truly powerful magic items. While a magician can produce any number of magic items that hold requirements at least 4 levels below their own -- a wizard is permitted only one masterpiece at each level of their progression. It is no surprise, therefore, that characters would be vastly interested in acquiring magic items produced by others that are even of near equivalence to the mightiest items that a character could produce. A character could plausibly bind 8 magic items, and yet they can only create one which is of their highest level of effect. Gaining powerful magic items from other sources is a virtual requirement of the powerful adventurer.}

So it is of no surprise that there is a brisk -- if insanely risky -- trade in magical equipment amongst the mighty. All the ingredients are there: characters are often left holding onto items that they can't use (for example: a third fire scimitar) and they are totally willing to exchange them for other items that they might want (magical teapots that change the weather or helmets that allow a man to see in all directions). And while the mutual benefit of such trades is not to be downplayed, it is similarly obvious that the benefits of betrayal in such arrangements are amazingly amazing. Killing people and taking their magical stuff is what adventurers do, so handing magic items back and forth in a seedy bar in a planar metropolis is an obviously dangerous undertaking.

\subsection{Tamerlain's Economy: The Murderocracy}
\vspace*{-8pt}
\quot{``The soldier may die, but he must receive his pay.''}

\desc{Let's say that you don't want to exchange goods and services for other goods and services at all. Well, it's medieval times baby, there's totally another option. See, if you \emph{kill} people by stabbing them in the face when they want to be paid for things, you \emph{don't have to pay for things}. Indeed, if you have a big enough pack of gnolls at your back, you don't have to pay anything to anyone except your own personal posse of gnolls.}

\desc{The disadvantages of this plan are obvious -- people get super pissed when they find out that you murdered their daughter because it was that or pay for a handful of radishes. But let's face it: if that old man can't do anything about it because you've \emph{got a pack of gnolls} -- then seriously what's he going to do? And while this sort of thing is often as not the source for an adventure hook (some guy comes to you and whines about how his whole family was killed by orcs/gnolls/your mom/ ogres/demons/or whatever and suddenly you have to strike a blow for great justice), it is also a cold harsh reality that everyone in D\&D land has to live with. Remember: noone has written \underline{The Rights of Man}. Heck, no one has even written \underline{Leviathan}. The fact that survivors of an attack may appeal to the better nature of adventurers is pretty much the only recompense that our gnoll posse might fear should they simply forcibly dispossess everyone in your village.}

\desc{So people who have something that the \emph{really powerful} people want are in a lot of danger. If a dirt farmer who does all of his bargaining in and around the turnip economy suddenly finds himself with a pile of rubies that's \emph{bad news}. It's not that there aren't people who would be willing to trade that farmer fine clothing, good food, and even minor magic items for those rubies -- there totally are. But a pile of rubies is just big enough that a Marilith might take time out of her busy schedule to teleport in and murder his whole family for them. And he's a dirt farmer -- there's no way he has the force needed to even \emph{pretend} to have the force needed to stop her from doing it. So if you have planar currencies or powerful artifacts, you can't trade them to innkeepers and prostitutes. You can't even give them away save to other powerful people and organizations.}

That doesn't mean that there isn't a peasant who runs around with a ring that casts \spell{charm person} once a day or there isn't a minor bandit chief who happens to have a magic sword. Those guys totally exist and they may well wander the lands trying to parlay their tiny piece of asymmetric power into something more. But the vast majority of these guys don't go on to become famous adventurers or dark lords -- they get their stuff taken away from them the first time they go head to head with someone with real power. Good or Evil, Lawful or Chaotic, \emph{noone} wants some idiot to be running around with a ring that \spell{charms} people -- because frankly that's the kind of dangerous and an accident waiting to happen. If you happen to be powerful and see some small fry running around with some magic -- your natural inclination is to take it from them. It doesn't matter what your alignment is, it doesn't matter if the guy with the wand of \spell{lightning bolt} is currently ``abusing'' it, the fact is that if you don't take magic items away from little fish one of your enemies will. There is no right to private property. Noone owns anything, they just hold on to it until someone takes it from them.

\subsection{Beelzebub's Economy: The Trade in Favors}
\vspace*{-8pt}
\quot{``I'm certain that there's something we can do to help you\ldots\ but eventually you'll have to help us.''}

\desc{Every transaction in D\&D land is essentially barter. People trade a cloth sack for a handful of peas, people trade an embroidered silken sack for a handful of silver, and people trade a powerful magical sack for a handful of raw power. But in any of these cases, the exchange is a one-time swap of goods that one person wants more for goods the other person desires. But there is no reason it has to work like that. Modern economies abstract all of the exchanges by creating ``money'' that is an arbitrary tally of how much goods and services one can expect society to deliver -- thereby allowing everyone to ``trade'' for whatever they want regardless of what they happen to produce. Nothing nearly that awesome exists anywhere in the myriad worlds of Dungeons and Dragons.}

\desc{What one \emph{can} see in heavy use is the trade in \emph{favors}. This is just like getting paid in money except that your money is only good with the guy who paid it to you. So you can see why people might be reluctant to sell you things for it. And yet despite the extremely obvious disadvantages of this system, it is in extremely wide use at every level of every economy. And the reason is because it's really convenient. There is no guaranty that a King will have anything you want right now when he needs you to kill the dragon that is plaguing his lands. In fact, with a dragon plaguing his lands, the King is probably in the worst possible position to pay you anything. But once the lands aren't on fire and taxes start rolling in, he can probably pay you quite handsomely. Heck, in two years or so his daughter will be marrying age and since she's just going to end up as an aristocrat unless she becomes the apprentice and cohort of a real adventurer\ldots}

Failing to pay one's debts can have disastrous consequences in D\&D land. We're talking ``sold to hobgoblin slavers'' levels of bad. Heck, this is a world in which you can seriously go into a court of law and present ``He needed killing'' as an excuse for premeditated homicide, so people who renege on their favors owed are in actual mortal danger. Of course, everyone is in mortal danger all the time because in D\&D land you actually can have land shark attacks in your home town -- so it isn't like there are any less people who flake on duties and favors. Of course, if people know you let favors slide they might be less likely to pull you out of the way of oncoming land sharks. Even in Chaotic areas, pissing off your neighbors is rarely a great plan.