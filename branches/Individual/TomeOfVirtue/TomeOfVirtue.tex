\documentclass[10pt]{report}

\usepackage{appendix}
\usepackage{floatflt}
\usepackage{fancyhdr}
\usepackage{textcomp}
\usepackage[usenames]{color}
\usepackage{isoent}    %\sfrac
%\usepackage{palatino} %font
\usepackage{newcent}   %font
\usepackage{sectsty} %custom section headings

\newcommand{\normalsections}{
\sectionfont{\noindent\rule{\textwidth}{0.015in}\\\nohang}
\subsectionfont{\noindent\rule{\textwidth}{0.005in}\\\nohang}
}

\newcommand{\columnsections}{
\sectionfont{\vspace*{-20pt}\noindent\rule{3.5in}{0.015in}\\\nohang}
\subsectionfont{\vspace*{-20pt}\noindent\rule{3.5in}{0.005in}\\\nohang}
}


\sectionfont{\noindent\rule{\textwidth}{0.015in}\\\nohang}
\subsectionfont{\noindent\rule{3.5in}{0.005in}\\\nohang}

\usepackage[Bjarne]{fncychap} %Canned chapter headings


\usepackage{multicol}
\usepackage[bookmarks=true,colorlinks,linkcolor=cyan,breaklinks]{hyperref}



%%%%Margins%%%
\topmargin 0pt
\advance \topmargin by -\headheight
\advance \topmargin by -\headsep
\textheight 8.9in
\oddsidemargin -0.25in
\evensidemargin \oddsidemargin
\textwidth 7in
\oddsidemargin -0.25in
%\setlength {\parindent} {0pt}



%%%Formatting%%%
\newcommand{\ability}[2]{\smallskip \noindent \textbf{#1} #2}
\newcommand{\shortability}[2]{\noindent\textbf{#1} #2\\}
\newcommand{\bolded}[1]{\noindent\textbf{#1}}
\newcommand{\itemability}[2]{\item \textbf{#1} #2}
\newcommand{\featname}[1]{\vspace*{0.1cm plus 0.2cm minus 0.05cm}\noindent\textbf{#1}\\}
\newcommand{\featnamelist}[1]{\vspace*{0.1cm plus 0.2cm minus 0.05cm}\noindent\textbf{#1}}

\newcommand{\descfeat}[2]{\featname{#1}\emph{#2}\\}

\newcommand{\classname}[1]{\subsection{#1}}
\newcommand{\condition}[1]{\emph{#1}}
\newcommand{\quot}[1]{\emph{#1}\medskip}
\newcommand{\desc}[1]{#1 \medskip}
\newcommand{\example}[1]{\emph{#1}}
\newcommand{\magicitem}[1]{\emph{#1}}
\newcommand{\monster}[1]{\subsection{#1} \label{monster:#1}}
\newcommand{\monsterline}[2]{\textbf{#1:} #2\\}
\newcommand{\monstersizetype}[2]{\textbf{#1 #2}\\}
\newcommand{\spell}[1]{\emph{#1}}
\newcommand{\spelllist}[1]{\smallskip \noindent \underline{\textbf{#1}}}

% For the feats -- moved here from feats.tex because we added another 
% section(s) for feats. -Surgo
\newcommand{\minitabular}[1]{\begin{tabular}{p{0.25in}p{2.9in}} #1\\ \end{tabular}}
\newcommand{\babfeat}[7]{
	\noindent\minitabular{\multicolumn{2}{l}{\parbox{3in}{\textbf{#1}}}}
	\minitabular{\multicolumn{2}{l}{\parbox{3in}{\small #2}}}
	\minitabular{\raggedleft\textbf{\small \textbf{+0:}}& {\small #3}}
	\minitabular{\raggedleft\textbf{\small +1:} & {\small #4}}
	\minitabular{\raggedleft\textbf{\small +6:} & {\small #5}}
	\minitabular{\raggedleft\textbf{\small +11:} & {\small #6}}
	\minitabular{\raggedleft\textbf{\small +16:} &{\small #7}}
}
\newcommand{\skillfeat}[8]{
	\noindent\minitabular{\multicolumn{2}{l}{\parbox{3in}{\textbf{#1}}}}
	\minitabular{\multicolumn{2}{l}{\parbox{3in}{\small #2}}}
	\minitabular{\multicolumn{2}{l}{\parbox{3in}{\small\textbf{#3}}}}
	\minitabular{\raggedleft\textbf{\small \textbf{0:}} &{\small #4}}
	\minitabular{\raggedleft\textbf{\small 4:} &{\small #5}}
	\minitabular{\raggedleft\textbf{\small 9:} &{\small #6}}
	\minitabular{\raggedleft\textbf{\small 14:} &{\small #7}}
	\minitabular{\raggedleft\textbf{\small 19:} &{\small #8}}
}

\newcommand{\boxit}[1]{\frame{\parbox{\textwidth}{#1}}}

% A new box command -- the first argument is the box's header, the second the 
% contents of the box. This looks a lot better than \boxit. -Surgo
\newcommand{\abox}[2]{\vspace{5pt}
	\fbox{\begin{minipage}{0.8\linewidth}
	\setlength{\parindent}{0.15in}
	\textbf{#1}

	#2
\end{minipage}}
\vspace{5pt}}

\newcommand{\itemspace}{\setlength{\itemsep}{-1mm}\setlength{\topsep}{-1mm} }

\newcommand{\listone}{\begin{list}{$\bullet$}{\itemspace}}
\newcommand{\listprereq}{\begin{list}{\vspace*{-2pt}}{\itemspace}}
\newcommand{\listtwo}{\begin{list}{$\triangleright$}{\itemspace}}
\newcommand{\listthree}{\begin{list}{--}{\itemspace}}

% Bold the first argument in the listitem
\newcommand{\bolditem}[2]{\item \textbf{#1} #2}

\newcommand{\slashfrac}[2]{${}^{#1}$\hspace*{-1pt}\makebox[2pt]{$\diagup$}${}_{#2}$}
%\newcommand{\half}[0]{\slashfrac{1}{2}}
\newcommand{\half}[0]{\ensuremath{\sfrac{1}{2}} }
\newcommand{\third}[0]{\ensuremath{\sfrac{1}{3}} }
\newcommand{\fourth}[0]{\ensuremath{\sfrac{1}{4}} }
%\newcommand{\half}[0]{\ensuremath{\scriptscriptstyle 1/2}}
%\newcommand{\threefourths}[0]{\ensuremath{\sfrac{3}{4}}}


%\rule{\textwidth}{0.005in}

%\renewcommand{\sectionmark}[1]{\markright{\thesection\ \boldmath\emph{#1}\unboldmath}\\\rule{\textwidth}{0.005in}\\}

\setcounter{tocdepth}{1}

\begin{document}

\pagestyle{plain}
%Cover Page

\begin{center} \Huge
\textsc{ClasslessD20 Modern}
\end{center}

\newpage


\pagestyle{fancy}
%Fix the spacing so it's all reasonable
\linespread{.9}  \small  \normalsize \itemspace \normalsections

\tableofcontents

\chapter{Alignment Discussion: Good, Evil. Fanaticism.}

\chapter{Compromise, Coexistence, and Conversion}

\chapter{Classes}

\input{baseclasses/paladin}
%\input{baseclasses/cleric}

\chapter{Bonus Core Classes}

\input{baseclasses/curator}
\input{baseclasses/marshal}
\input{baseclasses/sohei}
%\input{baseclasses/whitemage}

\chapter{Prestige Classes}

\input{prestige/holycrusader}

\chapter{Feats/Spells/Spheres}

\section{Feats}

\begin{multicols}{2}

\hypertarget{feat:weaponrighteous}{}\babfeat{Weapon of Righteous Destruction [Combat]}{

Your hands make whatever is being held by them holy and on fire. For some reason this doesn't make them melt or burn up.}{

Whatever weapon you are wielding is considered Magical (+1/3 bonus/level) in addition to any other properties that it has. Your unarmed attacks, even if not proficient, count for this effect.}{

The above, Flaming weapon.}{

The above, Holy instead of Flaming.}{

The above, Sun weapon, Fort save. (BoG)}{

The above, Vorpal weapon (BoG).}

The following feats from the Tome of Fiends are now also [Celestial] feats, and you don't pick up fiendish traits for taking them as a [Celestial] feat:

Breath Weapon 

Elemental Aura 

Extra Arms 

Extra Summons 

Greater Teleport 

Harmless Form 

Large Size 

Huge Size 

Heighten Spell Like Ability 

\descfeat{Product of Celestial Dalliance}{One of your recent ancestors was an celestial Outsider or from a good-aligned plane. Maybe your parents play it off as a virgin birth, maybe your dad became a Saint.}
You may take any [Celestial] feat, and gain Resistance 5 to Acid, Cold, Electricity. You also gain one of the Angel, Archon, Eladrin, or Guardinal subtypes, and a Smite Evil attack usable at will that does bonus damage equal to 1/2 of your strength modifier.

\featname{Wings of Good [Celestial]}
You gain wings and a fly speed equal to double your base land speed with good maneuverability that requires special armor to stay aloft. These must be feathery or energy-based.

\hypertarget{feat:lordofdeath}{}\skillfeat{Lord of Death [Necromantic] [Skill] [Leadership]}{
A whole bunch of skeletons and crap show up to fight under your tattered banner.}{
Knowledge (Religion) Ranks:}{
You have a Command Rating equal to your ranks in Knowledge Religion divided by five (round up). You are a Necromantic leader (see Heroes of Battle).}{
You can muster a group of followers. Your leadership score is your ranks in Knowledge Religion plus your Wisdom modifier. Your followers are all mindless Undead. You don't make them or anything, they just show up.}{
You are able to delegate command to a loyal cohort. Your cohort is an intelligent and loyal Undead creature with a CR at least 2 less than your character level. Cohorts gain levels when you do.}{
Your followers swell in number to that of an army.}{
Your allies gain energy resistance to Positive Energy equal to your level while they are within line of sight of you.}

\hypertarget{feat:tyrant}{}\skillfeat{Tyrant [Skill] [Leadership]}{
You push people around and get larger and larger groups trapped in the iron gauntlet of your brutal rule.}{
Intimidate Ranks:}{
You inspire such terror that creatures you intimidate continue to act intimidated after you leave, too afraid to raise their voice in defiance even after you have apparently left them far behind.}{
You can muster a group of followers. Your leadership score is your ranks in Intimidate plus your Strength modifier.}{
Your followers swell in number to that of an army.}{
Your mere presence inspires fear and can break a battle. Enemies with more than 5 hit dice less than you do must make a Will save (DC 10 + � Level + Strength Modifier) of flee in \emph{panic}.  This is a [Fear] effect.}{
Your presence causes despair in even brave opponents. All enemies within 30' of your suffer a -2 Morale penalty to Willpower saves.}

\hypertarget{feat:monsterrancher}{}\skillfeat{Monster Rancher [Skill] [Leadership]}{
You can breed and train a large number of crazy beasts.}{
Handle Animal Ranks:}{
You can use Handle Animal as if it were Diplomacy when dealing with Magical Beasts and Dragons. You can do similarly with Aberrations and Plants with an Intelligence Score that is less than 9.}{
You can muster a group of followers. Your leadership score is your ranks in Handle Animal plus any synergy bonuses you gt to that skill. Your followers can, and must be monsters.}{
You have a loyal cohort that is a monster of some kind. A cohort is an intelligent and loyal creature with a CR at least 2 less than your character level. Cohorts gain levels when you do.}{
You know what any monster is unless it is disguised by illusion, and you can look up its stat line in the appropriate monster book when devising your strategies.}{
Once per day, you can reroll a saving throw allowed by a Supernatural Ability.}

\hypertarget{feat:armyofdemons}{}\skillfeat{Army of Demons [Skill] [Fiend] [Celestial] [Leadership]}{
You have an army of planar crazy crap.}{
Knowledge (Planes) Ranks:}{
You have a Command Rating equal to your Knowledge: Planes ranks divided by five (round up).}{
You can muster a group of followers. Your leadership score is your ranks in Knowledge: Planes plus your Charisma mod. These followers can and must be outsiders.}{
Your followers swell in number to that of an army.}{
You own a planar stronghold.}{
Your allies gain a +2 morale bonus to all saving throws if they can see you and you are within medium range.}

\hypertarget{feat:bureaucrat}{}\skillfeat{Bureaucrat [Skill] [Leadership]}{
You have a functioning guild that makes stuff for you and gives you money.}{
Appraise Ranks:}{
You draw an income for working as an administrator, getting 1 GP/week per rank in Appraise.}{
You can muster a group of followers. Your leadership score is your ranks in Appraise plus your Intelligence modifier. These followers all have profession and craft skills.}{
You get your own Stronghold.}{
You get a +2 bonus to profit checks.}{
Your guild goes planar, your number of followers swell to the size of an army and their ranks start filling up with producers and managers from other planes of existence.}

\end{multicols}

\section{Spells}

\section{New Spells}

\subsection{Some Spells Don't Work}

Many spells are underwhelming for their level or have mechanics that are hard to explain. But first and foremost of all the spells that are bad for the game is Polymorph. That spell is integral to any fantasy setting, but people haven't made it work in 3rd edition. Mostly, this is because people keep writing it long instead of short. Remember, if you can't explain an effect in 2 minutes, everyone else is already confused.

\subsection{Necromancy}

Necromancy as a school is possessed of some of the most powerful and game defining spells ever imagined in the worlds of Dungeons and Dragons. \spell{Magic Jar}, \spell{Wail of the Banshee}, and \spell{Clone} can practically be a world threatening plan for a BBEG all by themselves. In fact, that's been done several times. But Necromancy as a school suffers greatly for this attention. Though the earth shaking power for dark lords is well represented, the low levels of necromancy have been largely ignored by generations of authors. It is our intention to produce a short list of spells that allow a low level Necromancer to be memorable and effective without constantly falling back on the old stand-by of having Spell Focus: Conjuration.

The [Healing] subschool:
The spell \spell{cure light wounds} has no business being in the school of Conjuration. It's not that you can't make an acceptable argument for the existence of ``conjuration" that makes people feel better -- that's actually pretty easy to rationalize. It's that \spell{cure light wounds} doesn't work the way a spell that was in Conjuration would work. It doesn't create healthy flesh to fill up wounds -- it channels Positive Energy into the creature and makes them feel better or worse depending upon how they react to that sort of thing. As described, the [Healing] subschool needs to be in the same school as \spell{inflict light wounds}, because it does the same thing. Logically speaking, that could be Evocation (because Evocation handles any Energy Channeling), or it could be Necromancy (because Necromancy can do pretty much anything with Positive or Negative Energy). We suggest having the [Healing] subschool in Necromancy, but only because this isn't The Tome of Evocation. If you decide to make these spells Evocation spells for your home game, we won't stop you.\\

\begin{quote}

\featname{Congealing Consumption}
\begin{small}
\shortability{Necromancy}{}
\shortability{Level:}{Sor/Wiz 2}
\shortability{Components:}{V, S}
\shortability{Casting Time:}{1 Standard Action}
\shortability{Range:}{Medium}
\shortability{Area:}{10 foot radius burst.}
\shortability{Duration:}{Instantaneous, and 1 round/level (see below)}
\shortability{Saving Throw:}{Willpower Negates}
\shortability{Spell Resistance:}{Yes.}
\end{small}
\emph{As the necromancer finishes the final incantations, a dark cloud arises and envelopes the souls of those within.}

Any creature within the area when the spell is cast must make a Willpower save or be nauseated for one round per level of the caster.\\

\featname{Curse of Crumbling Conviction}
\begin{small}
\shortability{Necromancy}{}
\shortability{Level:}{Sor/Wiz 4}
\shortability{Components:}{V, S}
\shortability{Casting Time:}{1 Standard Action}
\shortability{Range:}{Medium}
\shortability{Target:}{One Creature.}
\shortability{Duration:}{Instantaneous}
\shortability{Saving Throw:}{Willpower Negates}
\shortability{Spell Resistance:}{Yes.}
\end{small}
\emph{The avenging angel glared down with menace at the necromancer. She raised her flaming sword even as he completed his spell. He met her smoldering gaze levelly. ``Why?" he asked. It was a question she could not answer\ldots}

If the target fails their save they no longer feel strongly about people, ideals, or things. The target's alignment becomes neutral if it wasn't already, and the creature becomes indifferent to everyone, including the caster. This effect is an instantaneous shift, and in no way prevents a creature from responding to subsequent diplomacy or threats. This lack of purpose is an oppressive feeling for intelligent creatures, who will gladly adopt the alignment of whatever creature next persuades them to being helpful. Creatures with an alignment subtype will gradually find their purpose again -- regaining the alignment of their subtype in a d4 days unless they already have a new one.\\

\featname{Dark Symmetry}
\begin{small}
\shortability{Necromancy}{}
\shortability{Level:}{Sor/Wiz 2}
\shortability{Components:}{S}
\shortability{Casting Time:}{1 Standard Action}
\shortability{Range:}{Medium}
\shortability{Target:}{One Creature.}
\shortability{Duration:}{Concentration}
\shortability{Saving Throw:}{Willpower Negates}
\shortability{Spell Resistance:}{Yes.}
\end{small}
\emph{Her hands slow down from the frenzied pace and the necromancer's shadow extends into the target's. The warrior's sword arm slows and holds fast. A smile flashes across her face, and she takes a step forward. The warrior unsteadily takes a step backwards, a look of panic crossing his face.}

If the victim fails their saving throw, they are helpless and unable to voluntarily move until the spell is terminated. Further, if the caster moves while concentrating upon the spell, the victim simultaneously moves an equal distance in the same direction. If the victim is moved into an occupied space, he falls prone. If the victim is moved off a cliff, he falls.\\

\featname{Form of Death}
\begin{small}
\shortability{Necromancy}{}
\shortability{Level:}{Sor/Wiz 2}
\shortability{Components:}{S}
\shortability{Casting Time:}{1 Standard Action}
\shortability{Range:}{Touch}
\shortability{Target:}{One Living Creature.}
\shortability{Duration:}{24 hours}
\shortability{Saving Throw:}{Fortitude Negates (Harmless)}
\shortability{Spell Resistance:}{Yes (Harmless).}
\end{small}
\emph{The final incantations completed, the necromancer's skin turned gray, his lips became cold and dry.}

A creature affected by Form of Death becomes very much like an undead creature. The target gains [Undead] as a subtype and is affected by any spell or effect that targets Undead specifically. The target is also cured by negative energy and damaged by positive energy. The target is immune to negative energy levels, ability damage, and ability drain. The character will be treated as an undead by those around him, which is both a boon and a bane -- while mindless undead won't attack the target unless specifically ordered to and the character gains a +4 profane bonus on all charisma related checks when used on undead creatures, the target also suffers a -4 penalty on Charisma related checks for dealing with living creatures.\\

\featname{Puppet Dance}
\begin{small}
\shortability{Necromancy}{}
\shortability{Level:}{Sor/Wiz 3}
\shortability{Components:}{S}
\shortability{Casting Time:}{1 Standard Action}
\shortability{Range:}{Close}
\shortability{Target:}{One Corporeal Creature.}
\shortability{Duration:}{Concentration}
\shortability{Saving Throw:}{Reflex Negates}
\shortability{Spell Resistance:}{Yes.}
\end{small}
\emph{The necromancer holds her hands up with fingers apart, shadowy tendrils hang down from each finger, tapering into nonexistence before reaching her waist. As the spell reaches completion, larger tendrils appear above the target and hang down to anchor themselves in the victim's flesh.}

If the victim fails their saving throw, they are helpless and unable to move voluntarily until the spell is terminated. This spell only affects creatures with a physical body. When the caster spends a standard action to concentrate on the spell, she may opt to have the victim move and perform a physical standard action. The caster cannot force the victim to use their spell knowledge (if any), and any attacks made by the victim use the caster's Base Attack Bonus rather than their own.\\

\featname{Sobering Skeletal Stillness}
\begin{small}
\shortability{Necromancy}{}
\shortability{Level:}{Sor/Wiz 1}
\shortability{Components:}{V, S}
\shortability{Casting Time:}{1 Standard Action}
\shortability{Range:}{Medium}
\shortability{Target:}{One Creature with Bones.}
\shortability{Duration:}{Concentration}
\shortability{Saving Throw:}{Fortitude Negates}
\shortability{Spell Resistance:}{Yes.}
\end{small}
\emph{Chortling like a man possessed, the necromancer contorts his hands into unnatural positions, emitting dreadful crackling sounds of bones grinding against one another. A black aura surrounds his victim, and the sounds of crepitace now come from two\ldots}

If the victim fails their saving throw, they are helpless and unable to move until the spell is terminated. This spell only affects creatures who have a skeletal structure, although an exoskeleton does count. Creatures normally immune to paralysis, necromantic effects, or effects requiring a fortitude save that do not affect objects are still affected by this spell if they have a skeleton (so a zombie ogre is affected, but an iron golem is not).\\

\featname{Tasha's Tomb Tainting}
\begin{small}
\shortability{Necromancy}{}
\shortability{Level:}{Sor/Wiz 1}
\shortability{Components:}{V, S, M}
\shortability{Casting Time:}{10 minutes}
\shortability{Range:}{Touch}
\shortability{Area:}{40-ft. radius emanating from the touched point}
\shortability{Duration:}{Instantaneous}
\shortability{Saving Throw:}{See text}
\shortability{Spell Resistance:}{No.}
\end{small}
\emph{The graveyard has a serene feeling; the dead have lain here undisturbed for generations. Above the gate a sign in ancient Dwarven earnestly proclaims that the heroes interred within shall lie in peace forever. The necromancer puzzles through the faint and archaic runes and chuckles to himself. ``Not hardly?" he mutters, and begins fishing through his spell component pouch for a black pearl\ldots}

Upon spell completion, the area is free of any consecration, desecration, Forsaken Graveyard, hallow, Tomb, or unhallow effects. If the caster chooses, the area can be considered desecrated for the next 24 hours (the effects are increased as if there had been a permanent altar to an evil god or pantheon in the area).

\shortability{Material Component:}{One Black Pearl, worth at least 500 gp.}

\featname{Tasha's Tomb Transport}
\begin{small}
\featname{Necromancy}
\shortability{Level:}{Sor/Wiz 3}
\shortability{Components:}{V, S}
\shortability{Casting Time:}{1 minute}
\shortability{Range:}{See Text}
\shortability{Target:}{You and objects and willing creatures.}
\shortability{Duration:}{Instantaneous}
\shortability{Saving Throw:}{See text}
\shortability{Spell Resistance:}{No.}
\end{small}
\emph{The caster finishes the droning incantations and places her hand on the ground, where dark red runes appear in a circle around it. Eerily dark tendrils rise from the shadows and consume everyone and everything within the area. Far away, a black portal opens on the ground and the travelers rise from it covered in a thin sheen of cold sweat.}

This spell can only be cast in a Tomb, and it transports the targets to another Tomb of the caster's choice. The caster has no ability to determine where in the Tomb she will end up, but all targets appear together. The target Tomb need not be on the same plane of existence, but the caster must know where the target Tomb is to within one mile. The spell fails if either Tomb is cut off from the Negative Energy Plane (including effects like dimensional interdiction). Total transported creatures and objects cannot exceed 500 pounds per caster level in weight.\\

\featname{Tomb Tile Tessellation}
\begin{small}
\shortability{Necromancy}{}
\shortability{Level:}{Sor/Wiz 2}
\shortability{Components:}{V, S, M}
\shortability{Casting Time:}{10 minutes}
\shortability{Range:}{Touch}
\shortability{Area:}{One ten-foot cube}
\shortability{Duration:}{Instantaneous}
\shortability{Saving Throw:}{No.}
\shortability{Spell Resistance:}{No.}
\end{small}
\emph{Who knew how long the dead had lain in repose, unmolested and forgotten? Judging by the state of the remains, it was long enough -- so the necromancer brought out the charts and began planning out how to make sure it stayed that way.}

This spell can only be cast in a place where the dead have lain for at least 50 years without being returned as undead. Each casting makes one ten foot cube eligible to become a Tomb. Once the entire area is eligible to become a Tomb, the entire area becomes a Tomb.

\shortability{Material Component:}{One vial of Holy Water or one vial of Unholy Water.}
\end{quote}


\subsection{Polymorph}

\subsubsection{Polymorph Version 1: Character Replacement}

If you take part of your character -- any part of your character -- and part of a monster from one of the many monster books in D\&D, and you put them together into a single Voltron-like body, you have broken D\&D. That should be obvious, but since we are over six years into the ridiculous circus that is polymorph in 3rd edition Dungeons and Dragons, apparently it isn't. If it is important to you that you be allowed to dumpster dive through the monster books and find an appropriate to transform into, it is important to D\&D that absolutely no part of your character be mixed and matched during that period. If you want to truly become a monster, you have to actually become that monster. Not ``the monster with all my spell effects running", not ``the monster with my formidable mental attributes. No. You need to become the monster exactly as it appears in the monster book or there's no chance of you getting a balanced result. Some people are going to end up as mediocre monsters with carry-over abilities that happen to synergize well and become tremendously powerful while other people are just as unbalanced in the other direction when they find that drawbacks of their character are carried over and overwrite the abilities of a monster that are supposed to make them any good at all.

And this isn't just hyperbole or doomsday predictions, this is established fact. We've all played with some of the multitude of different versions of Polymorph errata and ``fixes", and the abject horror caused by every single iteration. The idea doesn't work. If you're going to replace any part of the character, you have to replace it all. So here's a version of polymorph that won't make us cry. This ain't rocket science, it just takes a little bit of discipline:

\begin{quote}
\featname{Polymorph Self}
\begin{small}
\shortability{Transmutation}{}
\shortability{Level:}{Sor/Wiz 4}
\shortability{Components:}{V, S}
\shortability{Casting Time:}{1 Standard Action}
\shortability{Range:}{Self}
\shortability{Duration:}{10 minutes/level (D)}
\shortability{Saving Throw:}{Fortitude Negates (Harmless)}
\shortability{Spell Resistance:}{No}
\end{small}
\quot{``A Turtle am I? Let's see how Turtlike I\ldots\  CAN\ldots\  BE!'' And with that, the mage was a giant turtle.}

You vanish and a monster of your choice appears in your place. The creature shares your alignment, personality and goals, and will continue to act as you would within the limits of its intelligence and abilities. The creature must be at least 3 CR less than your character level, may not have the incorporeal or swarm subtype, and is unexceptional for its type. If the monster is killed, the spell is ended. When the spell ends, the monster vanishes and you appear where the monster was with an amount of lethal, nonlethal, and ability damage on you equal to the amount the monster had suffered when the spell ended (this means that if the spell ended because the monster was slain and the monster had an equal or greater number of hit points as you, you may well be dead when you appear).
\end{quote}

\begin{quote}
\featname{Polymorph Other}
\begin{small}
\shortability{Transmutation}{}
\shortability{Level:}{Sor/Wiz 4}
\shortability{Components:}{V, S}
\shortability{Casting Time:}{1 Standard Action}
\shortability{Range:}{Medium}
\shortability{Target:}{One Creature}
\shortability{Duration:}{Permanent (D)}
\shortability{Saving Throw:}{Fortitude Negates}
\shortability{Spell Resistance:}{Yes}
\end{small}
\quot{The witch snarled at the trespasser and pointed her wand vindictively at him. A short incantation later left nothing but a pig in his place.}

Your target vanishes and a creature of your choice appears in its place. The creature shares the alignment, personality and goals of the target, and will continue to act as it would within the limits of its intelligence and abilities. The creature must be at least 5 CR less than your character level, may not have the incorporeal or swarm subtype, and is unexceptional for its type. If the creature is killed, the spell is ended. When the spell ends, the creature vanishes and the target appears where the creature was with an amount of lethal, nonlethal, and ability damage on it equal to the amount the creature had suffered when the spell ended (this means that if the spell ended because the creature was slain and the creature had an equal or greater number of hit points as the original target, it may well be dead when it appears).

\end{quote}


\begin{quote}
\featname{Mass Polymorph}
\begin{small}
\shortability{Transmutation}{}
\shortability{Level:}{Sor/Wiz 7}
\shortability{Components:}{V, S}
\shortability{Casting Time:}{1 Standard Action}
\shortability{Range:}{Medium}
\shortability{Target:}{Any number of creatures within a 20' radius}
\shortability{Duration:}{Permanent (D)}
\shortability{Saving Throw:}{Fortitude Negates}
\shortability{Spell Resistance:}{Yes}
\end{small}
\quot{The crowd looked uncomfortable. They had weapons and were brandishing them in a fashion quite menacing. But the magician was laughing, and that really put a damper on the mood of the entire event. They started to regain their composure and again advance upon him. He snorted and muttered an incantation, and something about swine \ldots}

Each target vanishes and creature of your choice appears in its place. The creatures share the alignment, personality and goals of the targets, and will continue to act as they would within the limits of their intelligence and abilities. The creatures must be at least 7 CR less than your character level, need not be the same for all targets, none may have the incorporeal or swarm subtype, and all are unexceptional for their type. If a creature is killed, the spell is ended for that target only. When the spell ends, the creatures vanish and the targets appear where the creatures were with an amount of lethal, nonlethal, and ability damage on it equal to the amount the creature had suffered when the spell ended (this means that if the spell ended for a target because the creature was slain and the creature had an equal or greater number of hit points as the original target, it may well be dead when it appears).

\end{quote}

\subsubsection{Polymorph Version 2: Fixed Forms}

The other version is one where transforming leaves you essentially yourself, only with a new hairdo and possibly some bonuses. In this case, you keep everything about yourself and simply get a disguise and some advantages consistent with a buff spell. All of the ``Whatever-Form" spells don't stack with multiple castings or even with each other, because they are considered to be ``one spell makes another spell irrelevant" for purposes of spell stacking.


\begin{quote}\featname{Human Form}
\begin{small}
\shortability{Transmutation}{}
\shortability{Level:}{Brd 1; Sor/Wiz 2}
\shortability{Components:}{V, S}
\shortability{Casting Time:}{1 Standard Action}
\shortability{Range:}{Touch}
\shortability{Target:}{One Willing Creature}
\shortability{Duration:}{10 minutes/level}
\shortability{Saving Throw:}{Fortitude Negates (Harmless)}
\shortability{Spell Resistance:}{Yes}
\end{small}
\quot{The man looked at the fallen prince and smiled. He whispered some eldritch words, and then there were two princes. One living, and one dead. The living prince smiled.}

The target assumes the appearance of a specific individual of medium size or smaller, or of a generic member of a humanoid race. The target is effectively disguised, and gains a +10 bonus on Disguise checks made to impersonate the genuine article. The target suffers no penalties to Disguise for assuming the visage of a different race or sex.
\end{quote}


\begin{quote}
\featname{Lycanthropy}
\begin{small}
\shortability{Transmutation}{}
\shortability{Level:}{Sor/Wiz 3}
\shortability{Components:}{V, S}
\shortability{Casting Time:}{1 Standard Action}
\shortability{Range:}{Touch}
\shortability{Target:}{One Willing Creature}
\shortability{Duration:}{10 minutes/level}
\shortability{Saving Throw:}{Fortitude Negates (Harmless)}
\shortability{Spell Resistance:}{Yes}
\end{small}
\quot{The shaman howled in rage and transformed into a wolverine.}

The target assumes the appearance of a specific or generic animal or magical beast of small, medium, or large size. The target is effectively disguised, and gains a +10 bonus on Disguise checks made to impersonate the genuine article. The target suffers no penalties to Disguise for assuming the visage of a different race or sex. The new form is unable to use normal equipment (all carried or worn items meld into the new form when the spell takes effect), and has whatever natural weapons the caster desires (to a maximum of 1 natural weapon per four levels). These natural weapons inflict an amount of damage appropriate for a magical beast of the new form's size. Any equipment the character had is subsumed into their new form.
\listone
    \item Small, Flying:90' flight speed (good), +4 Dex, -4 strength
    \item Small, Land:+2 Dex
    \item Small, Swimming:60' swim speed
    \item Medium, Flying:60' flight speed (good), +2 Dex
    \item Medium, Land:40' land speed, +2 Strength, +2 Natural Armor
    \item Medium, Swimming: 60' swim speed, +2 Strength, +2 Natural Armor
    \item Large, Flying: 90' flight speed (average), +2 Dex, +4 strength, +1 Natural Armor
    \item Large, Land: +6 Strength, +5 Natural Armor
    \item Large, Swimming: 60' swim speed, +6 Strength, +4 Natural Armor
\end{list}
\end{quote}


\begin{quote}
\featname{Monstrous Form}
\begin{small}
\shortability{Transmutation}{}
\shortability{Level:}{Sor/Wiz 4}
\shortability{Components:}{V, S}
\shortability{Casting Time:}{1 Standard Action}
\shortability{Range:}{Touch}
\shortability{Target:}{One Willing Creature}
\shortability{Duration:}{10 minutes/level (D)}
\shortability{Saving Throw:}{Fortitude Negates (Harmless)}
\shortability{Spell Resistance:}{Yes}
\end{small}
\quot{With a sweep of your cloak you become a creature of nightmare.}

The target assumes a horrific and monstrous countenance of a monster of Medium, Large, or Huge Size. The basic structure can look like pretty much anything, and the descriptions are just guidelines. All of the character's equipment melds into his new form. The character no longer has the ability to use equipment, but has a number of natural weapons appropriate to the new form:

\listone
    \item Yeth Hound (Medium):50' speed, +4 Str, +4 Dex, Bite, Improved Trip
    \item Displacer Beast (Large): +8 Str, +2 Dex, +5 Natural Armor, 1 Primary Bite and 2 secondary Tentacle Whips, Concealment.
    \item Monstrous Spider (Large): 30' Climb Speed, +8 Str, +8 Natural Armor Bonus, 1 natural weapon Bite, Poison (1d6 Con/ 1d6 Con)
    \item Chuul (Large): 60' Swim Speed, +8 Str, +6 Natural Armor, 2 Primary Pinchers, character gains the [Aquatic] Subtype.
    \item Bulette (Large): 20' Burrow Speed, +8 Str, +10 Natural Armor
    \item Manticore (Large): 60' Fly Speed (Average) +8 Str, +6 Natural Armor, 2 natural weapon Claws, 2 natural weapon ranged spikes attacks (1d8 + Str, 19-20 crit, 20' range increment)
    \item Giant Serpent (Huge): +14 Str, +10 Natural Armor Bonus, 1 natural weapon bite, Poison (1d6 Dex damage/1d6 Dex damage), Improved Grab.
\end{list}
\end{quote}


\begin{quote}\featname{Fiend Form}
\begin{small}
\shortability{Transmutation}{}
\shortability{Level:}{Sor/Wiz 5}
\shortability{Components:}{V, S}
\shortability{Casting Time:}{1 Standard Action}
\shortability{Range:}{Touch}
\shortability{Target:}{One Willing Creature}
\shortability{Duration:}{10 minutes/level (D)}
\shortability{Saving Throw:}{Fortitude Negates (Harmless)}
\shortability{Spell Resistance:}{Yes}
\end{small}
\quot{With a foul guttural utterance and a rude gesture, the wizard transforms into a fiend from the lower planes.}

The target assumes the appearance of a specific individual of medium size or smaller, or of a generic member of a fiendish race. The target is effectively disguised, and gains a +10 bonus on Disguise checks made to impersonate the genuine article. The target suffers no penalties to Disguise for assuming the visage of a different race or sex. While in Fiendish form, the target gains two bonus [Fiend] feats of your choice that it would meet the requirements for if it was actually a member of a fiendish race, and gains access to a sphere of your choice. In order to use a spell-like ability from the sphere, the target must expend one spell-slot or prepared spell of an equal or greater spell-level, but there is no other limit to how many times the spell-like abilities can be used. Rules for [Fiend] feats and spheres may be found in the Tome of Fiends.
\end{quote}


\begin{quote}\featname{Dragon Form}
\begin{small}
\shortability{Transmutation}{}
\shortability{Level:}{Sor/Wiz 6}
\shortability{Components:}{V, S}
\shortability{Casting Time:}{1 Standard Action}
\shortability{Range:}{Touch}
\shortability{Target:}{One Willing Creature}
\shortability{Duration:}{10 minutes/level (D)}
\shortability{Saving Throw:}{Fortitude Negates (Harmless)}
\shortability{Spell Resistance:}{Yes}
\end{small}
\quot{The final incantations are completed and you transform into a dragon.}

The target character assumes the form of a huge dragon. The character gets a +14 Strength bonus and a -4 Dexterity penalty. The character gains a +18 Natural Armor Bonus. The character gains immunity to one energy type (which must be Acid, Cold, Electricity, Fire, or Poison), and a breath weapon that inflicts 1d6 per level of the same type of damage. Using the breath weapon is a supernatural ability that requires a standard action and may only be used at most once every 1d4+1 rounds. The character has a flight speed of 120' with poor maneuverability. A character in Dragon Form has three natural attacks: a primary Bite and two secondary claws. Worn equipment is subsumed into the new draconic form.
\end{quote}

%\end{small}
%\end{multicols}

\subsection{Some Effects Don't work}

\subsubsection{Stacking Spell Resistance}

Spell Resistance does stack, but it does so in a really weird way that the authors have never actually taken the time to explain. SR is a DC for a level check, and that means that it is actually calculated as the number people are supposed to roll to penetrate your SR plus your CR. A SR of 15 on a CR 1 monster is awesome (it means that people of your level are supposed to roll a 14 to penetrate your SR), while a SR of 15 for a CR 13 monster is a joke (it means that enemies fail to penetrate your SR only on a natural 1). When you have Spell Resistance, and your CR goes up, your Spell Resistance also goes up. An Imp with a CR of 2 and a SR of 6 who takes enough levels of Wizard to gain a CR (2 levels as it happens) gains 1 SR as well and is SR 7.

But what happens if more than one source gives you SR? Well, it still stacks, it just does so in the aforementioned really weird way. First, you take your highest SR, then you start adding very small numbers to it based on what your other sources of SR would give you. If a secondary source of SR is less than 6 + your CR, having it increases your SR by +1. If a secondary source of SR is 6 + your CR or more, but less than 11 + our CR, your primary SR increases by +2. If a secondary source of SR is between 11 + CR and 15 + CR, it increases the primary SR by +3. And finally, a secondary source of SR that is 16 + CR or more adds +4 to the primary SR.

It would be nice if the basic rules ever explained that, but they don't. It doesn't come up all that often, but Drow Monks, for example, don't end up with SR in the high 30s at mid level. Their SR is actually just moderately impressive.

\subsubsection{Hiding in 3.5 D\&D is Dumb}

OK, we all know that it makes us feel kind of bad when the Rogue sneaks up on people and stabs them in the face without them ever seeing who did it. But you know what? People totally do that crap all the time. It's not even an uncommon occurrence, and there's really no cause to get excited about. The 3.5 rules for hiding, where you need cover or concealment to hide, are retarded. That makes Rogues run around with tower shields so that they can hide themselves and their equipment behind the cover of the tower shield (including the tower shield itself, which makes my brain hurt). Yes, you can totally hide when there are no intervening objects between you and the victim. It's called ``sneaking up behind people" and in a game with no facing it's handled with a hide check opposed by spot.\\

If you attempt to hide in a combat setting, you are under a number of restrictions:
\listone
    \item A character who has been attacked automatically can guess what square you are in. You may retain your invisibility, but that's just Full Concealment, and they could very plausibly hit you.
    \item There is a -20 penalty to Hide for attempting to fight while hidden. The distance penalties on Spot are pretty amazing, but most people can't hide at a -20 penalty.
    \item Once they see you, they see you. If an opponent successfully spots you even once (and they get to try every round while in combat), they just plain see you until you manage to get all the way out of their field of view (generally requiring you to leave the scene or make bluff checks or something).
    \item Spot Bonuses can get quite large. A spotter who knows what he's looking for gets a +4 bonus, and a spotter who is extremely familiar with the target gets a +10 bonus -- these bonuses are weirdly listed under the Disguise skill, but they still apply (so if someone says ``There's a halfling Ninja over there!" every other Guard gets a +4 bonus).
\end{list}

\vspace*{8pt}

But you can do it. Hiding in combat is hard, but it's a thing that powerful characters may be able to do against some opponents. Some of the D\&D authors have an outdated idea that Rogues should be forced to ``hide in shadows" or something. But this is D\&D, and most enemies have Darkvision. There are no shadows. Attempting to force Rogues to hide only in areas that they could plausibly hide in if a suspicious person was looking right at them and knew what they were looking for is incredibly cruel. In any kind of stressful situation that isn't an accurate picture of what is going on.

\section{Celestial Spheres}
\quot{``In brightest day, in darkest night, no evil shall escape my sight.''}

Celestial Spheres function like Fiendish Spheres, except that they are associated with not-Evil rather than not-Good. Any Fiendish Spheres which are not associated with Evil also qualify as Celestial Spheres.

\newcommand{\sphere}[9]{\begin{list}{}{\itemspace}\item \textbf{1:} \spell{#1} \item \textbf{3:} \spell{#2} \item \textbf{5:} \spell{#3} \item \textbf{7:} \spell{#4} \item \textbf{9:} \spell{#5} \item \textbf{11:} \spell{#6} \item \textbf{13:} \spell{#7} \item \textbf{15:} \spell{#8} \item \textbf{17:} \spell{#9}}
\newcommand{\spherecont}[1]{\item\textbf{19:} \spell{#1} \end{list}\medskip}

\begin{multicols}{2}
\section{Spheres} \hypertarget{spheres}{}

\newcommand{\sphere}[9]{\begin{list}{}{\itemspace}\item \textbf{1:} \spell{#1} \item \textbf{3:} \spell{#2} \item \textbf{5:} \spell{#3} \item \textbf{7:} \spell{#4} \item \textbf{9:} \spell{#5} \item \textbf{11:} \spell{#6} \item \textbf{13:} \spell{#7} \item \textbf{15:} \spell{#8} \item \textbf{17:} \spell{#9}}
\newcommand{\spherecont}[1]{\item\textbf{19:} \spell{#1} \end{list}\medskip}

Fiends, celestials, and some characters cast magic primarily through spell-like abilities. While many monsters and characters get arbitrary spell lists, spheres present a way to advance spellcasting in a thematic way. When a creature has access to a sphere, she is able to use all of the abilities within that sphere up to her character level. If they gains more levels, more powers of the sphere become available. In this way the spell-like abilities of fiends created with the rules in this tome should always be \ae sthetically and level appropriate.

\ability{Basic Sphere Access:}{When a creature has basic access to a sphere, she can use any of the spells listed in the sphere may be used once per day (each) as spell-like abilities, provided that their listed level is equal or lower to the creature's character level.}

\ability{Advanced Sphere Access:}{When a creature has advanced access to a sphere, she can use any of the spells listed in the sphere may be used 3 times per day (each) as spell-like abilities, provided that their listed level is equal or lower to the creature's character level.}

\ability{Expert Sphere Access:}{When a creature has expert access to a sphere, any spells listed in the sphere may be used at will as spell-like abilities, provided that their listed level is equal or lower to the creature's character level.}

\ability{Creating new spheres:}{The following list of spheres isn't intended to be comprehensive, and we fully expect that some players and DMs will want many more spheres than we have scribed. All new spheres must be approved of by the DM, and should represent some actual (indifferent or evil) trait like ``intoxication" or ``badgers" rather than a game mechanical notion like ``kicking ass and being totally sweet" or something praiseworthy like ``generosity". A good place to start is actually Domains, as these are already a source by which a character gain a spell at every odd-numbered level.}

\ability{Spheres and Spell Levels:}{Spell-like abilities used out of spheres are considered to be cast as a spell level equal to half the minimum needed character level to use the ability (rounded up). The save DC of a spell-like ability granted through Sphere access is Charisma-based. Thus, the save DC for a spell-like ability which becomes available at character level 5 is 13 + Charisma bonus.}
\end{multicols}

\chapter{It's a Miracle!}

Much like \spell{wish}, \spell{miracle} is a problematic spell that makes us sad. Not because it's open-ended in a way that encourages player creativity, we're all for that. It's because it is more accurately titled ``invoke DM fiat''. You are very specifically spending a 9th level spell slot and 5,000 experience points to \emph{trick the DM into breaking the game for you}. And that needs fixing just as much as something which explicitly lets you break the game does.

Our solution is to very explicitly say what a \spell{miracle} is capable of -- and not capable of. The amount of flexibility is still well within what we want from a \spell{miracle}, there are a lot of 7th level or lower spells that do all kinds of things, so the players -- and the DM -- have plenty of room for creativity without it devolving into a mandated Rule 0 fiasco. There's still some room for interpretation, but that's a matter of what's thematic rather than a matter of effect legitimacy.

\input{miracle}

\chapter{Celestial Superbeings}

\section{Angels}

\section{Exemplar: The Lawful, the Chaotic, and the Furry}

\chapter{Gods and Divinity}

\chapter{Revised Diplomancy Rules}

\chapter{Celestial and Precelestial Economics}

\section{Hope}

\chapter{High Adventure on the Upper Planes}

%\input{highadventure_upper}

\end{document}