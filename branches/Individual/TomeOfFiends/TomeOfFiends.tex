\documentclass[10pt]{report}

\usepackage{appendix}
\usepackage{floatflt}
\usepackage{fancyhdr}
\usepackage{textcomp}
\usepackage[usenames]{color}
\usepackage{isoent}    %\sfrac
%\usepackage{palatino} %font
\usepackage{newcent}   %font
\usepackage{sectsty} %custom section headings

\newcommand{\normalsections}{
\sectionfont{\noindent\rule{\textwidth}{0.015in}\\\nohang}
\subsectionfont{\noindent\rule{\textwidth}{0.005in}\\\nohang}
}

\newcommand{\columnsections}{
\sectionfont{\vspace*{-20pt}\noindent\rule{3.5in}{0.015in}\\\nohang}
\subsectionfont{\vspace*{-20pt}\noindent\rule{3.5in}{0.005in}\\\nohang}
}


\sectionfont{\noindent\rule{\textwidth}{0.015in}\\\nohang}
\subsectionfont{\noindent\rule{3.5in}{0.005in}\\\nohang}

\usepackage[Bjarne]{fncychap} %Canned chapter headings


\usepackage{multicol}
\usepackage[bookmarks=true,colorlinks,linkcolor=cyan,breaklinks]{hyperref}



%%%%Margins%%%
\topmargin 0pt
\advance \topmargin by -\headheight
\advance \topmargin by -\headsep
\textheight 8.9in
\oddsidemargin -0.25in
\evensidemargin \oddsidemargin
\textwidth 7in
\oddsidemargin -0.25in
%\setlength {\parindent} {0pt}



%%%Formatting%%%
\newcommand{\ability}[2]{\smallskip \noindent \textbf{#1} #2}
\newcommand{\shortability}[2]{\noindent\textbf{#1} #2\\}
\newcommand{\bolded}[1]{\noindent\textbf{#1}}
\newcommand{\itemability}[2]{\item \textbf{#1} #2}
\newcommand{\featname}[1]{\vspace*{0.1cm plus 0.2cm minus 0.05cm}\noindent\textbf{#1}\\}
\newcommand{\featnamelist}[1]{\vspace*{0.1cm plus 0.2cm minus 0.05cm}\noindent\textbf{#1}}

\newcommand{\descfeat}[2]{\featname{#1}\emph{#2}\\}

\newcommand{\business}[6]{
	\vspace{2pt}
	\noindent\hspace*{1cm}\textbf{#1}\\
	\noindent\hspace*{1cm}\textbf{Primary Skill:} #2\\
	\noindent\hspace*{1cm}\textbf{Secondary Skills:} #3\\
	\noindent\hspace*{1cm}\textbf{Capital:} #4\\
	\noindent\hspace*{1cm}\textbf{Resources:} #5\\
	\noindent\hspace*{1cm}\textbf{Risk:} #6
	\vspace{2pt}
}
\newcommand{\classname}[1]{\subsection{#1}}
\newcommand{\condition}[1]{\emph{#1}}
\newcommand{\quot}[1]{\emph{#1}\medskip}
\newcommand{\desc}[1]{#1 \medskip}
\newcommand{\example}[1]{\emph{#1}}
\newcommand{\magicitem}[1]{\emph{#1}}
\newcommand{\monster}[1]{\subsection{#1} \label{monster:#1}}
\newcommand{\monsterline}[2]{\textbf{#1:} #2\\}
\newcommand{\monstersizetype}[2]{\textbf{#1 #2}\\}
\newcommand{\spell}[1]{\emph{#1}}
\newcommand{\spelllist}[1]{\smallskip \noindent \underline{\textbf{#1}}}

% For the feats -- moved here from feats.tex because we added another 
% section(s) for feats. -Surgo
\newcommand{\minitabular}[1]{\begin{tabular}{p{0.25in}p{2.9in}} #1\\ \end{tabular}}
\newcommand{\babfeat}[7]{
	\noindent\minitabular{\multicolumn{2}{l}{\parbox{3in}{\textbf{#1}}}}
	\minitabular{\multicolumn{2}{l}{\parbox{3in}{\small #2}}}
	\minitabular{\raggedleft\textbf{\small \textbf{+0:}}& {\small #3}}
	\minitabular{\raggedleft\textbf{\small +1:} & {\small #4}}
	\minitabular{\raggedleft\textbf{\small +6:} & {\small #5}}
	\minitabular{\raggedleft\textbf{\small +11:} & {\small #6}}
	\minitabular{\raggedleft\textbf{\small +16:} &{\small #7}}
}
\newcommand{\skillfeat}[8]{
	\noindent\minitabular{\multicolumn{2}{l}{\parbox{3in}{\textbf{#1}}}}
	\minitabular{\multicolumn{2}{l}{\parbox{3in}{\small #2}}}
	\minitabular{\multicolumn{2}{l}{\parbox{3in}{\small\textbf{#3}}}}
	\minitabular{\raggedleft\textbf{\small \textbf{0:}} &{\small #4}}
	\minitabular{\raggedleft\textbf{\small 4:} &{\small #5}}
	\minitabular{\raggedleft\textbf{\small 9:} &{\small #6}}
	\minitabular{\raggedleft\textbf{\small 14:} &{\small #7}}
	\minitabular{\raggedleft\textbf{\small 19:} &{\small #8}}
}

\newcommand{\boxit}[1]{\frame{\parbox{\textwidth}{#1}}}

% A new box command -- the first argument is the box's header, the second the 
% contents of the box. This looks a lot better than \boxit. -Surgo
\newcommand{\abox}[2]{\vspace{5pt}
	\fbox{\begin{minipage}{0.8\linewidth}
	\setlength{\parindent}{0.15in}
	\textbf{#1}

	#2
\end{minipage}}
\vspace{5pt}}

\newcommand{\itemspace}{\setlength{\itemsep}{-1mm}\setlength{\topsep}{-1mm} }

\newcommand{\listone}{\begin{list}{$\bullet$}{\itemspace}}
\newcommand{\listprereq}{\begin{list}{\vspace*{-2pt}}{\itemspace}}
\newcommand{\listtwo}{\begin{list}{$\triangleright$}{\itemspace}}
\newcommand{\listthree}{\begin{list}{--}{\itemspace}}

% Bold the first argument in the listitem
\newcommand{\bolditem}[2]{\item \textbf{#1} #2}

\newcommand{\slashfrac}[2]{${}^{#1}$\hspace*{-1pt}\makebox[2pt]{$\diagup$}${}_{#2}$}
%\newcommand{\half}[0]{\slashfrac{1}{2}}
\newcommand{\half}[0]{\ensuremath{\sfrac{1}{2}} }
\newcommand{\third}[0]{\ensuremath{\sfrac{1}{3}} }
\newcommand{\fourth}[0]{\ensuremath{\sfrac{1}{4}} }
%\newcommand{\half}[0]{\ensuremath{\scriptscriptstyle 1/2}}
%\newcommand{\threefourths}[0]{\ensuremath{\sfrac{3}{4}}}


%\rule{\textwidth}{0.005in}

%\renewcommand{\sectionmark}[1]{\markright{\thesection\ \boldmath\emph{#1}\unboldmath}\\\rule{\textwidth}{0.005in}\\}

\setcounter{tocdepth}{1}

\begin{document}

\pagestyle{plain}
%Cover Page

\begin{center} \Huge
\textsc{ClasslessD20 Modern}
\end{center}

\newpage


\pagestyle{fancy}
%Fix the spacing so it's all reasonable
\linespread{.9}  \small  \normalsize \itemspace \normalsections

\tableofcontents

\chapter{Morality and Fiends}

\input{alignment}

\chapter{Fiends with Class}

\section{Base Classes}

What follows are four base classes. The Summoner is a magician class available to any player character race, while the other three are intended only for Fiendish use and are of primary utility to construct and advance Fiends for use as villains and cohorts.

\input{baseclasses/truefiend}
\input{baseclasses/fiendishbrute}
\input{baseclasses/conduit}
\input{baseclasses/summoner}

\section{Prestige Classes}

For something that has received so much ink, the world of fiend related prestige classes is remarkably non-functional. Fiendish Cultists, Dark Summoners, and Fiend-blooded Sorcerers are classic D\&D fodder. But unfortunately, the previously published classes for these archetypes are generally� not good. And that makes us sad. Here are some classes designed to fill those perceived holes:

\input{prestige/hellwalker}
\input{prestige/seeroftempest}
\input{prestige/initiateofblacktower}
\input{prestige/barrister}
\input{prestige/celestialbeacon}
\input{prestige/boatman}

\chapter{Fiends With Style}

This section contains new feats and spheres as well as an optional rule to make people assuming the mantle of fiendish power seem a little more� fiendish.

\section{The Feats}

\begin{multicols}{2}

\hypertarget{feat:weaponrighteous}{}\babfeat{Weapon of Righteous Destruction [Combat]}{

Your hands make whatever is being held by them holy and on fire. For some reason this doesn't make them melt or burn up.}{

Whatever weapon you are wielding is considered Magical (+1/3 bonus/level) in addition to any other properties that it has. Your unarmed attacks, even if not proficient, count for this effect.}{

The above, Flaming weapon.}{

The above, Holy instead of Flaming.}{

The above, Sun weapon, Fort save. (BoG)}{

The above, Vorpal weapon (BoG).}

The following feats from the Tome of Fiends are now also [Celestial] feats, and you don't pick up fiendish traits for taking them as a [Celestial] feat:

Breath Weapon 

Elemental Aura 

Extra Arms 

Extra Summons 

Greater Teleport 

Harmless Form 

Large Size 

Huge Size 

Heighten Spell Like Ability 

\descfeat{Product of Celestial Dalliance}{One of your recent ancestors was an celestial Outsider or from a good-aligned plane. Maybe your parents play it off as a virgin birth, maybe your dad became a Saint.}
You may take any [Celestial] feat, and gain Resistance 5 to Acid, Cold, Electricity. You also gain one of the Angel, Archon, Eladrin, or Guardinal subtypes, and a Smite Evil attack usable at will that does bonus damage equal to 1/2 of your strength modifier.

\featname{Wings of Good [Celestial]}
You gain wings and a fly speed equal to double your base land speed with good maneuverability that requires special armor to stay aloft. These must be feathery or energy-based.

\hypertarget{feat:lordofdeath}{}\skillfeat{Lord of Death [Necromantic] [Skill] [Leadership]}{
A whole bunch of skeletons and crap show up to fight under your tattered banner.}{
Knowledge (Religion) Ranks:}{
You have a Command Rating equal to your ranks in Knowledge Religion divided by five (round up). You are a Necromantic leader (see Heroes of Battle).}{
You can muster a group of followers. Your leadership score is your ranks in Knowledge Religion plus your Wisdom modifier. Your followers are all mindless Undead. You don't make them or anything, they just show up.}{
You are able to delegate command to a loyal cohort. Your cohort is an intelligent and loyal Undead creature with a CR at least 2 less than your character level. Cohorts gain levels when you do.}{
Your followers swell in number to that of an army.}{
Your allies gain energy resistance to Positive Energy equal to your level while they are within line of sight of you.}

\hypertarget{feat:tyrant}{}\skillfeat{Tyrant [Skill] [Leadership]}{
You push people around and get larger and larger groups trapped in the iron gauntlet of your brutal rule.}{
Intimidate Ranks:}{
You inspire such terror that creatures you intimidate continue to act intimidated after you leave, too afraid to raise their voice in defiance even after you have apparently left them far behind.}{
You can muster a group of followers. Your leadership score is your ranks in Intimidate plus your Strength modifier.}{
Your followers swell in number to that of an army.}{
Your mere presence inspires fear and can break a battle. Enemies with more than 5 hit dice less than you do must make a Will save (DC 10 + � Level + Strength Modifier) of flee in \emph{panic}.  This is a [Fear] effect.}{
Your presence causes despair in even brave opponents. All enemies within 30' of your suffer a -2 Morale penalty to Willpower saves.}

\hypertarget{feat:monsterrancher}{}\skillfeat{Monster Rancher [Skill] [Leadership]}{
You can breed and train a large number of crazy beasts.}{
Handle Animal Ranks:}{
You can use Handle Animal as if it were Diplomacy when dealing with Magical Beasts and Dragons. You can do similarly with Aberrations and Plants with an Intelligence Score that is less than 9.}{
You can muster a group of followers. Your leadership score is your ranks in Handle Animal plus any synergy bonuses you gt to that skill. Your followers can, and must be monsters.}{
You have a loyal cohort that is a monster of some kind. A cohort is an intelligent and loyal creature with a CR at least 2 less than your character level. Cohorts gain levels when you do.}{
You know what any monster is unless it is disguised by illusion, and you can look up its stat line in the appropriate monster book when devising your strategies.}{
Once per day, you can reroll a saving throw allowed by a Supernatural Ability.}

\hypertarget{feat:armyofdemons}{}\skillfeat{Army of Demons [Skill] [Fiend] [Celestial] [Leadership]}{
You have an army of planar crazy crap.}{
Knowledge (Planes) Ranks:}{
You have a Command Rating equal to your Knowledge: Planes ranks divided by five (round up).}{
You can muster a group of followers. Your leadership score is your ranks in Knowledge: Planes plus your Charisma mod. These followers can and must be outsiders.}{
Your followers swell in number to that of an army.}{
You own a planar stronghold.}{
Your allies gain a +2 morale bonus to all saving throws if they can see you and you are within medium range.}

\hypertarget{feat:bureaucrat}{}\skillfeat{Bureaucrat [Skill] [Leadership]}{
You have a functioning guild that makes stuff for you and gives you money.}{
Appraise Ranks:}{
You draw an income for working as an administrator, getting 1 GP/week per rank in Appraise.}{
You can muster a group of followers. Your leadership score is your ranks in Appraise plus your Intelligence modifier. These followers all have profession and craft skills.}{
You get your own Stronghold.}{
You get a +2 bonus to profit checks.}{
Your guild goes planar, your number of followers swell to the size of an army and their ranks start filling up with producers and managers from other planes of existence.}

\end{multicols}

\subsection{The Spheres}

Fiends (and some of their minions and associates) cast magic primarily through spell-like abilities. While many signature fiends have arbitrary lists of spell-like abilities, the Tome of Fiends offers a method to advance Fiends into thematically appropriate spell-like abilities when they advance. When a fiend has access to a sphere, she is able to use all of the abilities within that sphere up to her character level. If she gains more levels, more powers of the sphere become available. In this way the spell-like abilities of fiends created with the rules in this tome should always be \ae sthetically and level appropriate.

\ability{Basic Sphere Access:}{When a creature has basic access to a sphere, she can use any of the spells listed in the sphere may be used once per day (each) as spell-like abilities, provided that their listed level is equal or lower to the creature's character level.}

\ability{Advanced Sphere Access:}{When a creature has advanced access to a sphere, she can use any of the spells listed in the sphere may be used 3 times per day (each) as spell-like abilities, provided that their listed level is equal or lower to the creature's character level.}

\ability{Expert Sphere Access:}{When a creature has expert access to a sphere, any spells listed in the sphere may be used at will as spell-like abilities, provided that their listed level is equal or lower to the creature's character level.}

\ability{Creating new spheres:}{The following list of spheres isn't intended to be comprehensive, and we fully expect that some players and DMs will want many more spheres than we have scribed. All new spheres must be approved of by the DM, and should represent some actual (indifferent or evil) trait like ``intoxication" or ``badgers" rather than a game mechanical notion like ``kicking ass and being totally sweet" or something praiseworthy like ``generosity". A good place to start is actually Domains, as these are already a source by which a character gain a spell at every odd-numbered level.}

\ability{Spheres and Spell Levels:}{Spell-like abilities used out of spheres are considered to be cast as a spell level equal to half the minimum needed character level to use the ability (rounded up). The save DC of a spell-like ability granted through Sphere access is Charisma-based. Thus, the save DC for a spell-like ability which becomes available at character level 5 is 13 + Charisma bonus.}



\newcommand{\sphere}[9]{\begin{list}{}{\itemspace}\item \textbf{1:} \spell{#1} \item \textbf{3:} \spell{#2} \item \textbf{5:} \spell{#3} \item \textbf{7:} \spell{#4} \item \textbf{9:} \spell{#5} \item \textbf{11:} \spell{#6} \item \textbf{13:} \spell{#7} \item \textbf{15:} \spell{#8} \item \textbf{17:} \spell{#9}}
\newcommand{\spherecont}[1]{\item\textbf{19:} \spell{#1} \end{list}\medskip}

\begin{multicols}{2}
\section{Spheres} \hypertarget{spheres}{}

\newcommand{\sphere}[9]{\begin{list}{}{\itemspace}\item \textbf{1:} \spell{#1} \item \textbf{3:} \spell{#2} \item \textbf{5:} \spell{#3} \item \textbf{7:} \spell{#4} \item \textbf{9:} \spell{#5} \item \textbf{11:} \spell{#6} \item \textbf{13:} \spell{#7} \item \textbf{15:} \spell{#8} \item \textbf{17:} \spell{#9}}
\newcommand{\spherecont}[1]{\item\textbf{19:} \spell{#1} \end{list}\medskip}

Fiends, celestials, and some characters cast magic primarily through spell-like abilities. While many monsters and characters get arbitrary spell lists, spheres present a way to advance spellcasting in a thematic way. When a creature has access to a sphere, she is able to use all of the abilities within that sphere up to her character level. If they gains more levels, more powers of the sphere become available. In this way the spell-like abilities of fiends created with the rules in this tome should always be \ae sthetically and level appropriate.

\ability{Basic Sphere Access:}{When a creature has basic access to a sphere, she can use any of the spells listed in the sphere may be used once per day (each) as spell-like abilities, provided that their listed level is equal or lower to the creature's character level.}

\ability{Advanced Sphere Access:}{When a creature has advanced access to a sphere, she can use any of the spells listed in the sphere may be used 3 times per day (each) as spell-like abilities, provided that their listed level is equal or lower to the creature's character level.}

\ability{Expert Sphere Access:}{When a creature has expert access to a sphere, any spells listed in the sphere may be used at will as spell-like abilities, provided that their listed level is equal or lower to the creature's character level.}

\ability{Creating new spheres:}{The following list of spheres isn't intended to be comprehensive, and we fully expect that some players and DMs will want many more spheres than we have scribed. All new spheres must be approved of by the DM, and should represent some actual (indifferent or evil) trait like ``intoxication" or ``badgers" rather than a game mechanical notion like ``kicking ass and being totally sweet" or something praiseworthy like ``generosity". A good place to start is actually Domains, as these are already a source by which a character gain a spell at every odd-numbered level.}

\ability{Spheres and Spell Levels:}{Spell-like abilities used out of spheres are considered to be cast as a spell level equal to half the minimum needed character level to use the ability (rounded up). The save DC of a spell-like ability granted through Sphere access is Charisma-based. Thus, the save DC for a spell-like ability which becomes available at character level 5 is 13 + Charisma bonus.}
\end{multicols}

\input{fiendishrules}

\chapter{High Adventure on the Cordant Planes}
\quot{``Then you shall face your punishment in a realm of neither the Heavens nor the Hells...''}

The Cordant Planes are extremely different in character from the Upper and Lower Planes, just as the difference between Law and Chaos is fundamentally different from that between Good and Evil. Out of the Tomes, the Book of Gears deals most heavily with one of the Cordant Planes -- Mechanus -- so really this is the best place to handle them.

\section{High Adventure in... Mechanus!}

Mechanus is the plane of Ultimate Law, and while the actual nature of Law is dependent on how you handle Alignment in general, Mechanus is always giant clockwork land filled with robots and ant-people. Currently the official ``robots'' are the Inevitables, but you might very well have Modrons. The below examples assume

%\input{adventuremechanus}

\section{High Adventure in... The Outlands!}

The souls who end up in The Outlands are those who are ``Neutral'' or follow one of the dieties who live here. Given how alignments in 3.0 work, this means basically everyone who doesn't choose a particular other alignment and it \emph{also} includes druids and other ``balance nuts.'' The Outlands is a lot like the Prime Material Plane in that there is a bunch of nature and also a bunch of cities, except all of it is made of soul-stuff and magic cuts off near the center. Generally you'll be here either because you're traveling between the other planes via the gate-towns on its rim or because you have business with the Mind Flayers, Naga, Obad-hai, or any other of the random domains which exist around here.

%\input{adventureoutlands}

\section{High Adventure in... Limbo!}

Limbo is a land of Chaos, and like with the rest of the Law-Chaos fiasco it's not entirely clear what this means. Mechanically speaking, Limbo is made of a bunch of ``Chaos Matter'' which is for some reason shaped entirely by the minds around it and otherwise randomly cycles between the basic elements. Also the natives are all supernatural Giant Frogs or Githzerai. Despite the unstable nature of everything in Limbo, there are a few areas of interest which are either inhabited (and thus kept in a stable form pretty much all the time) or some sort of psychic storm.

%\input{adventurelimbo}

\input{ruleinhell}

\end{document}