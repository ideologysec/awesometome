\section{Illusion Magic: I Don't Believe This Crap}

\desc{Illusion magic has the distinguishing characteristic of being either the most powerful school of magic, or the least -- entirely at the whims of your playgroup. Illusions can be used as distractions, threats, enticements, concealment, modes of communication, prisons, attacks, disguises, false targets, entertainments, misdirections, religious inspiration, incitements to riot, madness provokers, commercial fraud, redecoration, time wasters, limited-use ability wasters (like prepared spells, scroll spells, or use-per-day spell-like abilities), or traps (in conjunction with dangerous terrain, monsters, substances, events, or magical effects). And that's just using the 1st level spell \spell{silent image}.}

\desc{People just don't expect their senses to lead them wrong, even in a world where people know that illusions exist. I mean, if a wall of fire suddenly pops up out of nowhere, it's actually more likely to actually be a real damaging wall made out of magical fire than it is to be an illusion of the same thing. And truthfully, who wants to pop a hand in to check? Not me either.}

\desc{What this means is that illusions are incredibly powerful because they allow such perfect forgeries of the real world. The downside of this is that lots of DMs try to counter the efforts of creative players by using a particularly harsh interpretation of the Disbelief rules in order to nerf illusions out of existence. It works like this: by the rules, you get a Will save vs. an illusion if you ``interact'' with it. DMs looking to throw salt in an illusionist's game usually allow that to mean ``in the same square as an illusion'' or ``looking at it.'' You also automatically make a save if you have ``proof that an illusion isn't real.'' What that means is anyone's guess, because in D\&D even the most unlikely circumstances could quite plausibly occur without illusionary influence. A silent orc moving through the grass might be a \spell{silent image} of an orc, an orc in a \spell{silence} effect, an incorporeal orc, or just an orc who happens to be really sneaky. Once you disbelieved the illusion, you suddenly got to see through its like it was transparent.}

\desc{Usually, DMs looking to punish illusionists will give multiple saves per turn, and then at some point just say that the target has automatically disbelieved the illusion, and this is possible only because the rules regarding illusions were written in the style of previous editions of D\&D called ``Rule 0'' where playing a pick-up game of D\&D involved a few hours of discussion about how the DM handled most effects. The current edition of D\&D (3.X) mostly did away with this because it sucks up valuable game time to have arguments about D\&D rules and it was the worst part of playing the game; however, illusions were never fully overhauled, so we are still stuck with this noise.}

\desc{Potential effects of illusions are also hotly debated. Some genius at WotC has laid down the law and said that the various \spell{image} and \spell{illusion} spells don't cause darkness, but that doesn't stop them from creating opaque mist or smoke or dust, obscuring objects, or even autumn leaves that drift around a person's head and float away from his touch, effectively blinding a person from dangers as well as complete darkness. Additionally, there are DM vs Player wars where DMs try to interpret the ``single object, creature, or force'' line to mean ``no more than one person or a monster in the illusion'' and players respond with things like ``its an illusion of a single force that summoned many monsters like the spell \spell{summon monster} or \spell{gate}'' or ``its one object connected by many invisible threads.'' Other DMs and players are convinced that you control all visual information in the Area of Effect, while others agree but say things like ``you can't trap a creature in a bubble with visual information on the inside that mimics the world except for some key creatures/object/terrain/effects, but people outside see him as normal because his image is on the outside of bubble.''}

In the end, it's a mess because the current rules can be made to do amazing things by creative people, but those amazing things break the level system and that means that DMs are forced to punish players for their creativity, thus hurting everyone. That being said, here are some playable rules regarding illusions that won't cause you to stab out your own eyes.